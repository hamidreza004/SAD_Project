
%----------------------------------------------------------------------------------------
%	PART
%----------------------------------------------------------------------------------------


\chapter{
	شرایط رضایت‌مندی
}

%----------------------------------------------------------------------------------------
%	CHAPTER 8
%----------------------------------------------------------------------------------------

%\chapterimage{chapter_head_2.pdf} % Chapter heading image

\section{معیارهای موفقیت}
در این قسمت، معیارهایی را مشخص می‌کنیم که تعیین‌کننده‌ی این هستند که پروژه با موفقیت به اتمام رسیده یا خیر. کنترل و نظارت بر این موارد در مراحل مختلف پروژه، به‌عهده‌ی مدیر پروژه است.

% \subsection{قابل‌قبول بودن سامانه‌ی اطلاعاتی حاصل برای کارفرما و کاربران}

\subsection{برخوردار بودن از کیفیت مناسب}
پس از به اتمام رسیدن پروژه، معرفی و ارائه‌ی آن به کارفرما باید رضایت‌مندی آن‌ها را به‌همراه داشته باشد و  تمام نیازمندی‌های مورد نظر در سامانه در نظر گرفته شده باشند.
یعنی سامانه‌ی حاصل باید از نظر معیارهایی که کارفرما برای شرکت تبیین کرده بود، مناسب و با کیفیت محسوب شود.
برای حاصل شدن این نتیجه، در پایان هر فاز و مرحله از پروژه، می‌توان نتیجه‌ی حاصل را برای کارفرما توصیف کرد و نظرات آن‌ها را در مراحل بعدی اعمال کرد تا نتیجه‌ی نهایی، رضایت‌مندی آن‌ها را به‌همراه داشته‌باشد.

همین‌طور در صورتی که کاربران بتوانند به‌آسانی از سامانه استفاده کنند و از آن نفع ببرند می‌توان گفت سامانه‌ی حاصل موفق بوده است. در این صورت سامانه نیاز‌ها و خواسته‌های آن‌ها را برطرف کرده و همین‌طور کاربرپسند است. در این صورت  است که می‌توان گفت از کیفیت مورد نظر کاربران نیز برخوردار است.

% چرا که در این صورت است که از سامانه‌ی حاصل استفاده شده و می‌توان گفت موفق است. همین‌طور اگر سامانه‌ی ما موفق تلقی شود و کاربران از آن راضی باشند، شرکت‌ها و سامانه‌های دیگر را کنار گذاشته و به‌سراغ سامانه‌ی ما می‌آیند. 

\subsection{اتمام پروژه در زمان مقرر}
می‌دانیم در بسیاری از پروژه‌ها، زمان مولفه‌ی بسیار مهمی است. به‌عنوان مثال در صورتی که  این پروژه به‌تاخیر بیافتد، بدین معنی است که شرکت‌ها و دانشجویانی که به‌امید راه‌اندازی این سامانه، مانند گذشته مراحل استخدام و یا کاریابی را دنبال نکرده‌اند، ممکن است متضرر شوند. مثلاً شرکت توسعه‌ی نرم‌افزاری که برای پروژه‌ای نیاز به یک طراح بازی داشته، ممکن است مانند همیشه برای این‌کار اعلامیه‌ای پخش نکرده باشد، و یا به‌اندازه‌ی کافی تبلیغ نکرده باشد. همچنین کارفرما ممکن است پروژه را برای بازه‌ی زمانی خاصی از ما بخواهد و ممکن است در صورت به‌تاخیر افتادن اتمام پروژه، ضرر زیادی به آن‌ها وارد گردد. 

بنابراین برنامه‌ریزی زمانی برای پروژه از حساس‌ترین مولفه‌های یک پروژه است. چرا که در صورتی که پروژه در زمان مقرر به اتمام نرسد، حتی می‌توان گفت شکست خورده است. بنابراین کار تخمین و نظارت بر مراحل مختلف پروژه بسیار مهم هستند. 

\subsection{اتمام پروژه با بودجه‌ی مشخص}
می‌دانیم به هر پروژه‌ای،از ابتدا بودجه‌ی مشخصی داده می‌شود. که آن بودجه از تخمین تمام هزینه‌ها به‌دست آمده و به کارفرما اعلام شده است. بنابراین قابل تغییر نیست. حال در صورتی که در طول پروژه بیشتر از این مقدار معین استفاده کنیم، باعث ضرر اقتصادی ما خواهد شد. بنابراین یکی از معیارها کنترل بودجه است. 

در این‌جا، بودجه‌ای به این پروژه اختصاص داده‌نشده است. بنابراین اگر خرجی داشته باشیم، باعث شکست پروژه‌ی ما خواهد شد. 

\section{پیش‌فرض‌ها}
تعدادی از پیش‌فرض‌های پروژه، به‌عنوان شرایط کارفرما در این‌جا ذکر شده‌اند. 
\begin{itemize}
	\item
	کارجویان همگی دانش‌جو و یا دانش‌آموخته‌ی دانشگاه صنعتی شریف هستند.
	\item
	اطلاعات وارد شده توسط کاربران معتبر هستند. به‌عنوان مثال آدرس وارد شده توسط کارفرمایان آدرسی معتبر و موجود خواهد بود. 
	\item 
	ادمین وبسایت به اطلاعات کاربران دسترسی خواهد داشت. 
	\item 
	فعال بودن این سامانه از لحاظ حقوقی بلامانع است.
\end{itemize}

\section{ریسک‌ها}

\begin{itemize}
	\item 
	تغییر فرضیات و نیازمندی‌های کارفرما در طول زمان
	
	راه‌کار:‌ در این صورت، با تحلیل نیازمندی‌ها، زمان و بودجه‌ی لازم به کارفرما اعلام خواهند شد.
	\item
	کناره‌گیری یکی از اعضای تیم
	
	راه‌کار: افزایش زمان بقیه اعضای تیم و یا در صورت نداشتن توانایی آن عضو اضافه کردن فرد دیگری به تیم.
	\item 
	رونمایی از سامانه‌ای مشابه <<شریف‌کار>>
	
	راه‌کار:‌ با بررسی خصوصیات و امکانات سامانه‌ی ارائه‌شده، سعی می‌کنیم از کاستی‌های آن استفاده کرده و به بهترین نحو از اطلاعات به‌دست آمده‌ی تعامل کاربران با آن سامانه استفاده کنیم تا سامانه‌ی خود را بهبود بخشیم. 
\end{itemize}

%\section{Paragraphs of Text}\index{Paragraphs of Text}
%
%\lipsum[1-7] % Dummy text
%
%%------------------------------------------------
%
%\section{Citation}\index{Citation}
%
%This statement requires citation \cite{article_key}; this one is more specific \cite[162]{book_key}.
%
%%------------------------------------------------
%
%\section{Lists}\index{Lists}
%
%Lists are useful to present information in a concise and/or ordered way\footnote{Footnote example...}.
%
%\subsection{Numbered List}\index{Lists!Numbered List}
%
%\begin{enumerate}
%	\item The first item
%	\item The second item
%	\item The third item
%\end{enumerate}
%
%\subsection{Bullet Points}\index{Lists!Bullet Points}
%
%\begin{itemize}
%	\item The first item
%	\item The second item
%	\item The third item
%\end{itemize}
%
%\subsection{Descriptions and Definitions}\index{Lists!Descriptions and Definitions}
%
%\begin{description}
%	\item[Name] Description
%	\item[Word] Definition
%	\item[Comment] Elaboration
%\end{description}
%
%%----------------------------------------------------------------------------------------
%%	CHAPTER 2
%%----------------------------------------------------------------------------------------
%
%\chapter{In-text Elements}
%
%\section{Theorems}\index{Theorems}
%
%This is an example of theorems.
%
%\subsection{Several equations}\index{Theorems!Several Equations}
%This is a theorem consisting of several equations.
%
%\begin{theorem}[Name of the theorem]
%	In $E=\mathbb{R}^n$ all norms are equivalent. It has the properties:
%	\begin{align}
%	& \big| ||\mathbf{x}|| - ||\mathbf{y}|| \big|\leq || \mathbf{x}- \mathbf{y}||\\
%	&  ||\sum_{i=1}^n\mathbf{x}_i||\leq \sum_{i=1}^n||\mathbf{x}_i||\quad\text{where $n$ is a finite integer}
%	\end{align}
%\end{theorem}
%
%\subsection{Single Line}\index{Theorems!Single Line}
%This is a theorem consisting of just one line.
%
%\begin{theorem}
%	A set $\mathcal{D}(G)$ in dense in $L^2(G)$, $|\cdot|_0$. 
%\end{theorem}
%
%%------------------------------------------------
%
%\section{Definitions}\index{Definitions}
%
%This is an example of a definition. A definition could be mathematical or it could define a concept.
%
%\begin{definition}[Definition name]
%	Given a vector space $E$, a norm on $E$ is an application, denoted $||\cdot||$, $E$ in $\mathbb{R}^+=[0,+\infty[$ such that:
%	\begin{align}
%	& ||\mathbf{x}||=0\ \Rightarrow\ \mathbf{x}=\mathbf{0}\\
%	& ||\lambda \mathbf{x}||=|\lambda|\cdot ||\mathbf{x}||\\
%	& ||\mathbf{x}+\mathbf{y}||\leq ||\mathbf{x}||+||\mathbf{y}||
%	\end{align}
%\end{definition}
%
%%------------------------------------------------
%
%\section{Notations}\index{Notations}
%
%\begin{notation}
%	Given an open subset $G$ of $\mathbb{R}^n$, the set of functions $\varphi$ are:
%	\begin{enumerate}
%		\item Bounded support $G$;
%		\item Infinitely differentiable;
%	\end{enumerate}
%	a vector space is denoted by $\mathcal{D}(G)$. 
%\end{notation}
%
%%------------------------------------------------
%
%\section{Remarks}\index{Remarks}
%
%This is an example of a remark.
%
%\begin{remark}
%	The concepts presented here are now in conventional employment in mathematics. Vector spaces are taken over the field $\mathbb{K}=\mathbb{R}$, however, established properties are easily extended to $\mathbb{K}=\mathbb{C}$.
%\end{remark}
%
%%------------------------------------------------
%
%\section{Corollaries}\index{Corollaries}
%
%This is an example of a corollary.
%
%\begin{corollary}[Corollary name]
%	The concepts presented here are now in conventional employment in mathematics. Vector spaces are taken over the field $\mathbb{K}=\mathbb{R}$, however, established properties are easily extended to $\mathbb{K}=\mathbb{C}$.
%\end{corollary}
%
%%------------------------------------------------
%
%\section{Propositions}\index{Propositions}
%
%This is an example of propositions.
%
%\subsection{Several equations}\index{Propositions!Several Equations}
%
%\begin{proposition}[Proposition name]
%	It has the properties:
%	\begin{align}
%	& \big| ||\mathbf{x}|| - ||\mathbf{y}|| \big|\leq || \mathbf{x}- \mathbf{y}||\\
%	&  ||\sum_{i=1}^n\mathbf{x}_i||\leq \sum_{i=1}^n||\mathbf{x}_i||\quad\text{where $n$ is a finite integer}
%	\end{align}
%\end{proposition}
%
%\subsection{Single Line}\index{Propositions!Single Line}
%
%\begin{proposition} 
%	Let $f,g\in L^2(G)$; if $\forall \varphi\in\mathcal{D}(G)$, $(f,\varphi)_0=(g,\varphi)_0$ then $f = g$. 
%\end{proposition}
%
%%------------------------------------------------
%
%\section{Examples}\index{Examples}
%
%This is an example of examples.
%
%\subsection{Equation and Text}\index{Examples!Equation and Text}
%
%\begin{example}
%	Let $G=\{x\in\mathbb{R}^2:|x|<3\}$ and denoted by: $x^0=(1,1)$; consider the function:
%	\begin{equation}
%	f(x)=\left\{\begin{aligned} & \mathrm{e}^{|x|} & & \text{si $|x-x^0|\leq 1/2$}\\
%	& 0 & & \text{si $|x-x^0|> 1/2$}\end{aligned}\right.
%	\end{equation}
%	The function $f$ has bounded support, we can take $A=\{x\in\mathbb{R}^2:|x-x^0|\leq 1/2+\epsilon\}$ for all $\epsilon\in\intoo{0}{5/2-\sqrt{2}}$.
%\end{example}
%
%\subsection{Paragraph of Text}\index{Examples!Paragraph of Text}
%
%\begin{example}[Example name]
%	\lipsum[2]
%\end{example}
%
%%------------------------------------------------
%
%\section{Exercises}\index{Exercises}
%
%This is an example of an exercise.
%
%\begin{exercise}
%	This is a good place to ask a question to test learning progress or further cement ideas into students' minds.
%\end{exercise}
%
%%------------------------------------------------
%
%\section{Problems}\index{Problems}
%
%\begin{problem}
%	What is the average airspeed velocity of an unladen swallow?
%\end{problem}
%
%%------------------------------------------------
%
%\section{Vocabulary}\index{Vocabulary}
%
%Define a word to improve a students' vocabulary.
%
%\begin{vocabulary}[Word]
%	Definition of word.
%\end{vocabulary}