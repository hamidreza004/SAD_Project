
%----------------------------------------------------------------------------------------
%	PART
%----------------------------------------------------------------------------------------


\chapter{
	شرایط رضایت‌مندی
}

%----------------------------------------------------------------------------------------
%	CHAPTER 8
%----------------------------------------------------------------------------------------

%\chapterimage{chapter_head_2.pdf} % Chapter heading image

\section{معیارهای موفقیت}
معیار‌های موفقیت مشخص می‌کند که آیا پروژه‌ی انجام شده موفق است یا خیر و لازم است که به صورت مستمر توسط مدیر پروژه کنترل و ارزیابی شود.
در ادامه معیار‌های موفقیت برای پروژه‌ی تعریف شده را مشخص می‌کنیم.

% \subsection{قابل‌قبول بودن سامانه‌ی اطلاعاتی حاصل برای کارفرما و کاربران}
\subsection{برخورداری از تمام ویژگی‌های مورد نظر}
نتیجه نهایی باید تمامی قابلیت‌هایی که کارفرما مشخص کرده است را داشته باشد و به درستی کار بکند. در صورت نقص در عملکرد یا نداشتن برخی قابلیت‌ها، پروژه موفقیت آمیز نبوده است.
همچنین این ویژگی‌ها باید با کیفیت مورد نظر کارفرما و با در نظر گرفتن به‌روش‌ها در تجربه کاربری پیاده شده باشد تا نتیجه از دید کاربر هم مطلوب باشد.

با اتمام هر بخش از پروژه می‌توان پروژه‌را به کارفرما ارائه کرد تا با اعمال نظرات  فرآیند انجام پروژه بهینه تر بوده و نتیجه به مطلوب کارفرما نزدیک تر باشد..

\subsection{اتمام پروژه در زمان مقرر}
اتمام پروژه در زمان مشخص شده از اهمیت به سزایی برخوردار است. چون ذینفع‌ها با تاخیر در انجام پروژه ممکن است متضرر شوند.
برای مثال ممکن است کاربران به این سیستم برای حسابداری شخصی نیاز داشته باشند و زمانی که تاخیر در کار نهایی وجود داشته باشند دچار مشکل شوند.
از طرفی مدیران نیز که در این کار سرمایه‌گذاری کرده‌اند لازم است تا برای بردن فضای رقابتی هر چه سریع‌تر این سیستم آماده باشد و با تاخیر ریسک سرمایه‌گذاری آن‌ها افزایش پیدا می‌کند.

پیش‌بینی و برنامه ریزی دقیق برای رسیدن به تمامی وظایف و انجام پروژه به صورت موفقیت آمیز امری لازم است. با تخمین مجدد و از بین بردن نقص‌های برنامه ریزی به صورت مستمر توسط مدیر پروژه می‌توان در زمان مشخص شده نتیجه مطلوب را به دست آورد.

\subsection{اتمام پروژه با بودجه‌ی مشخص}
از ابتدا برای پروژه بودجه تخصیص داده است و غیر قابل تغییر است. اگر بیشتر از این مقدار لازم باشد باعث فشار روی اعضای تیم انجام دهنده پروژه و همچنین افزایش ریسک یا ضرر برای کارفرما خواهد بود که معمولا به شکست خوردن پروژه می‌انجامد.

با توجه به اینکه این پروژه بودجه‌ای اختصاص داده نشده است، با حداقل منابع باید به نتیجه مطلوب رسید.

\section{پیش‌فرض‌ها}
تعدادی از پیش‌فرض‌های پروژه، به‌عنوان شرایط کارفرما در این‌جا ذکر شده‌اند. 
\begin{itemize}
	\item
	پروژه توسط دانشجویان درس تحلیل و طراحی سیستم‌ها در دانشگاه صنعتی شریف انجام شود.
	\item
	آدرس ایمیل و شماره کاربران تایید شده باشد.
	\item 
	ادمین به اطلاعات کاربران دسترسی داشته باشد. 
	\item 
	فعال بودن این سامانه از لحاظ حقوقی بلامانع است.
\end{itemize}

\section{ریسک‌ها}

\begin{itemize}
	\item 
	تغییر فرضیات و نیازمندی‌های کارفرما در طول زمان
	
	راه‌کار:‌ در این صورت، با تحلیل نیازمندی‌ها، زمان و بودجه‌ی لازم به کارفرما اعلام خواهند شد و در صورت توافق طرفین انجام می‌شود.
	\item
	کناره‌گیری یکی از اعضای تیم
	
	راه‌کار: این مشکل باید از سمت تیم انجام دهنده‌ی پروژه حل شود و در صورت نیاز زمان بقیه اعضا اضافه شده یا جایگزین شود.
	\item 
	به وجود آمدن رقابت برای دنگ
	
	راه‌کار:‌ با توافق کارفرما و تیم انجام دهنده تغییرات لازم برای رقابت اعمال شده و بودجه و نیرو با توجه به شرایط موجود بازنگری شده و در جهت رقابت و ایجاد سیستم بهتر و وظایف انجام داده شوند. 
\end{itemize}

%\section{Paragraphs of Text}\index{Paragraphs of Text}
%
%\lipsum[1-7] % Dummy text
%
%%------------------------------------------------
%
%\section{Citation}\index{Citation}
%
%This statement requires citation \cite{article_key}; this one is more specific \cite[162]{book_key}.
%
%%------------------------------------------------
%
%\section{Lists}\index{Lists}
%
%Lists are useful to present information in a concise and/or ordered way\footnote{Footnote example...}.
%
%\subsection{Numbered List}\index{Lists!Numbered List}
%
%\begin{enumerate}
%	\item The first item
%	\item The second item
%	\item The third item
%\end{enumerate}
%
%\subsection{Bullet Points}\index{Lists!Bullet Points}
%
%\begin{itemize}
%	\item The first item
%	\item The second item
%	\item The third item
%\end{itemize}
%
%\subsection{Descriptions and Definitions}\index{Lists!Descriptions and Definitions}
%
%\begin{description}
%	\item[Name] Description
%	\item[Word] Definition
%	\item[Comment] Elaboration
%\end{description}
%
%%----------------------------------------------------------------------------------------
%%	CHAPTER 2
%%----------------------------------------------------------------------------------------
%
%\chapter{In-text Elements}
%
%\section{Theorems}\index{Theorems}
%
%This is an example of theorems.
%
%\subsection{Several equations}\index{Theorems!Several Equations}
%This is a theorem consisting of several equations.
%
%\begin{theorem}[Name of the theorem]
%	In $E=\mathbb{R}^n$ all norms are equivalent. It has the properties:
%	\begin{align}
%	& \big| ||\mathbf{x}|| - ||\mathbf{y}|| \big|\leq || \mathbf{x}- \mathbf{y}||\\
%	&  ||\sum_{i=1}^n\mathbf{x}_i||\leq \sum_{i=1}^n||\mathbf{x}_i||\quad\text{where $n$ is a finite integer}
%	\end{align}
%\end{theorem}
%
%\subsection{Single Line}\index{Theorems!Single Line}
%This is a theorem consisting of just one line.
%
%\begin{theorem}
%	A set $\mathcal{D}(G)$ in dense in $L^2(G)$, $|\cdot|_0$. 
%\end{theorem}
%
%%------------------------------------------------
%
%\section{Definitions}\index{Definitions}
%
%This is an example of a definition. A definition could be mathematical or it could define a concept.
%
%\begin{definition}[Definition name]
%	Given a vector space $E$, a norm on $E$ is an application, denoted $||\cdot||$, $E$ in $\mathbb{R}^+=[0,+\infty[$ such that:
%	\begin{align}
%	& ||\mathbf{x}||=0\ \Rightarrow\ \mathbf{x}=\mathbf{0}\\
%	& ||\lambda \mathbf{x}||=|\lambda|\cdot ||\mathbf{x}||\\
%	& ||\mathbf{x}+\mathbf{y}||\leq ||\mathbf{x}||+||\mathbf{y}||
%	\end{align}
%\end{definition}
%
%%------------------------------------------------
%
%\section{Notations}\index{Notations}
%
%\begin{notation}
%	Given an open subset $G$ of $\mathbb{R}^n$, the set of functions $\varphi$ are:
%	\begin{enumerate}
%		\item Bounded support $G$;
%		\item Infinitely differentiable;
%	\end{enumerate}
%	a vector space is denoted by $\mathcal{D}(G)$. 
%\end{notation}
%
%%------------------------------------------------
%
%\section{Remarks}\index{Remarks}
%
%This is an example of a remark.
%
%\begin{remark}
%	The concepts presented here are now in conventional employment in mathematics. Vector spaces are taken over the field $\mathbb{K}=\mathbb{R}$, however, established properties are easily extended to $\mathbb{K}=\mathbb{C}$.
%\end{remark}
%
%%------------------------------------------------
%
%\section{Corollaries}\index{Corollaries}
%
%This is an example of a corollary.
%
%\begin{corollary}[Corollary name]
%	The concepts presented here are now in conventional employment in mathematics. Vector spaces are taken over the field $\mathbb{K}=\mathbb{R}$, however, established properties are easily extended to $\mathbb{K}=\mathbb{C}$.
%\end{corollary}
%
%%------------------------------------------------
%
%\section{Propositions}\index{Propositions}
%
%This is an example of propositions.
%
%\subsection{Several equations}\index{Propositions!Several Equations}
%
%\begin{proposition}[Proposition name]
%	It has the properties:
%	\begin{align}
%	& \big| ||\mathbf{x}|| - ||\mathbf{y}|| \big|\leq || \mathbf{x}- \mathbf{y}||\\
%	&  ||\sum_{i=1}^n\mathbf{x}_i||\leq \sum_{i=1}^n||\mathbf{x}_i||\quad\text{where $n$ is a finite integer}
%	\end{align}
%\end{proposition}
%
%\subsection{Single Line}\index{Propositions!Single Line}
%
%\begin{proposition} 
%	Let $f,g\in L^2(G)$; if $\forall \varphi\in\mathcal{D}(G)$, $(f,\varphi)_0=(g,\varphi)_0$ then $f = g$. 
%\end{proposition}
%
%%------------------------------------------------
%
%\section{Examples}\index{Examples}
%
%This is an example of examples.
%
%\subsection{Equation and Text}\index{Examples!Equation and Text}
%
%\begin{example}
%	Let $G=\{x\in\mathbb{R}^2:|x|<3\}$ and denoted by: $x^0=(1,1)$; consider the function:
%	\begin{equation}
%	f(x)=\left\{\begin{aligned} & \mathrm{e}^{|x|} & & \text{si $|x-x^0|\leq 1/2$}\\
%	& 0 & & \text{si $|x-x^0|> 1/2$}\end{aligned}\right.
%	\end{equation}
%	The function $f$ has bounded support, we can take $A=\{x\in\mathbb{R}^2:|x-x^0|\leq 1/2+\epsilon\}$ for all $\epsilon\in\intoo{0}{5/2-\sqrt{2}}$.
%\end{example}
%
%\subsection{Paragraph of Text}\index{Examples!Paragraph of Text}
%
%\begin{example}[Example name]
%	\lipsum[2]
%\end{example}
%
%%------------------------------------------------
%
%\section{Exercises}\index{Exercises}
%
%This is an example of an exercise.
%
%\begin{exercise}
%	This is a good place to ask a question to test learning progress or further cement ideas into students' minds.
%\end{exercise}
%
%%------------------------------------------------
%
%\section{Problems}\index{Problems}
%
%\begin{problem}
%	What is the average airspeed velocity of an unladen swallow?
%\end{problem}
%
%%------------------------------------------------
%
%\section{Vocabulary}\index{Vocabulary}
%
%Define a word to improve a students' vocabulary.
%
%\begin{vocabulary}[Word]
%	Definition of word.
%\end{vocabulary}
