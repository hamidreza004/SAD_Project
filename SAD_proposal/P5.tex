\chapter{رهیافت مدیریت}
\section{مدیریت، تجارب و وظایف}

\subsection{مدیر پروژه}
\begin{table}[H]
	\centering
	\resizebox{\textwidth}{!}{%
		\begin{tabular}{|C{0.2\textwidth}C{0.2\textwidth}C{0.2\textwidth}C{0.2\textwidth}C{0.2\textwidth}|}
			\hline
			\rowcolor[HTML]{F36619} 
			\multicolumn{5}{|c|}{\cellcolor[HTML]{fea82f}{\color[HTML]{FFFFFF}
					\Large
					\textbf{
						مدیر پروژه
					}
			}}                                   \\ \hline
			\rowcolor[HTML]{ffc288} 
			نام و نام خانوادگی
			&               
			مدرک تحصیلی    
			&                
			سابقه‌ی کاری
			&     
			شرح سوابق کاری
			&
			توانمندی‌ها
			\\
			\rowcolor[HTML]{fcecdd} 
			حمیدرضا هدایتی
			&               
			کارشناسی مهندسی کامپیوتر
			&    
			۱ سال
			&              
			مدیر پروژه در مجموعه هزاردستان - ستون	
			&
			مدیریت پروژه
			\\
			\hline
		\end{tabular}
	}
\end{table}



\subsection{اسکرام مستر}
\begin{table}[H]
	\centering
	\resizebox{\textwidth}{!}{%
		\begin{tabular}{|C{0.2\textwidth}C{0.2\textwidth}C{0.2\textwidth}C{0.2\textwidth}C{0.2\textwidth}|}
			\hline
			\rowcolor[HTML]{F36619} 
			\multicolumn{5}{|c|}{\cellcolor[HTML]{fea82f}{\color[HTML]{FFFFFF}
					\Large
					\textbf{
						مدیر پروژه
					}
			}}                                   \\ \hline
			\rowcolor[HTML]{ffc288} 
			نام و نام خانوادگی
			&               
			مدرک تحصیلی    
			&                
			سابقه‌ی کاری
			&     
			شرح سوابق کاری
			&
			توانمندی‌ها
			\\
			\rowcolor[HTML]{fcecdd} 
			ایمان غلامی
			&               
			کارشناسی مهندسی کامپیوتر
			&    
			۳ سال
			&              
			تیم لیدر در مجموعه هزاردستان - بلد	
			&
			رهبری تیم
			\\
			\hline
		\end{tabular}
	}
\end{table}

\subsection{اعضا و سوابق و توانمندی‌ها}
\begin{table}[H]
	\centering
	\resizebox{\textwidth}{!}{%
		\begin{tabular}{|C{0.2\textwidth}C{0.2\textwidth}C{0.2\textwidth}C{0.2\textwidth}C{0.2\textwidth}|}
			\hline
			\rowcolor[HTML]{F36619} 
			\multicolumn{5}{|c|}{\cellcolor[HTML]{fea82f}{\color[HTML]{FFFFFF}
					\Large
					\textbf{
						اعضا
					}
			}}                                   \\ \hline
			\rowcolor[HTML]{ffc288} 
			نام و نام خانوادگی
			&               
			مدرک تحصیلی    
			&                
			سابقه‌ی کاری
			&     
			شرح سوابق کاری
			&
			توانمندی‌ها
			\\
		\rowcolor[HTML]{fcecdd} 
			ایمان غلامی
			&               
			کارشناسی مهندسی کامپیوتر
			&    
			۳ سال
			&              
			تیم لیدر در مجموعه هزاردستان - بلد
			
			&
			\lr{Django, 
				Docker, Infrastructure, React}
			\\
			\rowcolor[HTML]{fcecdd} 
			حمیدرضا هدایتی
			&               
			کارشناسی مهندسی کامپیوتر
			&    
			۱ سال
			&              
			مدیر پروژه در مجموعه هزاردستان - ستون
			&
			\lr{
				Django, GoLang, Backend Development
			}
			\\
			\rowcolor[HTML]{fcecdd} 
			آرمین سعادت
			&               
			کارشناسی مهندسی کامپیوتر
			&    
			۵ ماه
			&              
			عضو فنی گروه آنالیسور
			&
			\lr{
	MongoDB,
	Node.js, 
	Database Design,
	Backend Development
}
			\\
			\hline
		\end{tabular}
	}
\end{table}

\subsection{تحلیل‌گران}
\begin{table}[H]
	\centering
	\resizebox{\textwidth}{!}{%
		\begin{tabular}{|C{0.25\textwidth}C{0.25\textwidth}C{0.25\textwidth}C{0.25\textwidth}|}
			\hline
			\rowcolor[HTML]{F36619} 
			\multicolumn{4}{|c|}{\cellcolor[HTML]{fea82f}{\color[HTML]{FFFFFF}
					\Large
					\textbf{
						تحلیل‌گران
					}
			}}                                   \\ \hline
			\rowcolor[HTML]{ffc288} 
			نام و نام خانوادگی
			&               
			مدرک تحصیلی    
			&                
			سابقه‌ی کاری
			&     
			شرح سوابق کاری
			\\
			\rowcolor[HTML]{fcecdd} 
			ایمان غلامی
			&               
			کارشناسی مهندسی کامپیوتر
			&    
			۳ سال
			&              
			تیم لیدر در مجموعه هزاردستان - بلد
			\\
			\rowcolor[HTML]{fcecdd} 
			حمیدرضا هدایتی
			&               
			کارشناسی مهندسی کامپیوتر
			&    
			۱ سال
			&              
			مدیر پروژه در مجموعه هزاردستان - ستون
			\\
			\hline
		\end{tabular}
	}
\end{table}

\subsection{طراح پایگاه‌داده}
\begin{table}[H]
	\centering
	\resizebox{\textwidth}{!}{%
		\begin{tabular}{|C{0.25\textwidth}C{0.25\textwidth}C{0.25\textwidth}C{0.25\textwidth}|}
			\hline
			\rowcolor[HTML]{F36619} 
			\multicolumn{4}{|c|}{\cellcolor[HTML]{fea82f}{\color[HTML]{FFFFFF}
					\Large
					\textbf{
					طراح پایگاه‌داده
					}
			}}                                   \\ \hline
			\rowcolor[HTML]{ffc288} 
			نام و نام خانوادگی
			&               
			مدرک تحصیلی    
			&                
			سابقه‌ی کاری
			&     
			شرح سوابق کاری
			\\
			\rowcolor[HTML]{fcecdd} 
			آرمین سعادت
			&               
			کارشناسی مهندسی کامپیوتر
			&    
			۵ ماه
			&              
			طراحی پایگاه‌داده و برنامه‌نویسی
			\lr{back-end}
			\\
			\hline
		\end{tabular}
	}
\end{table}


\subsection{طراحی گرافیکی و طراحی صفحات}
\begin{table}[H]
	\centering
	\resizebox{\textwidth}{!}{%
		\begin{tabular}{|C{0.25\textwidth}C{0.25\textwidth}C{0.25\textwidth}C{0.25\textwidth}|}
			\hline
			\rowcolor[HTML]{F36619} 
			\multicolumn{4}{|c|}{\cellcolor[HTML]{fea82f}{\color[HTML]{FFFFFF}
					\Large
					\textbf{
						طراحی گرافیکی و طراحی صفحات
					}
			}}                                   \\ \hline
			\rowcolor[HTML]{ffc288} 
			نام و نام خانوادگی
			&               
			مدرک تحصیلی    
			&                
			سابقه‌ی کاری
			&     
			شرح سوابق کاری
			\\
			\rowcolor[HTML]{fcecdd} 
			ایمان غلامی
			&               
			کارشناسی مهندسی کامپیوتر
			&    
			۲ ماه
			&              
			طراحی صفحات وب و برنامه‌نویسی 
			\lr{front-end, Vue.js}
			\\
						\rowcolor[HTML]{fcecdd} 
			آرمین سعادت
			&               
			کارشناسی مهندسی کامپیوتر
			&    
			دو ماه
			&              
			سابقه‌ی طراحی صفحات وب و کار با 
			\lr{Vue.js}
			\\
			\hline
		\end{tabular}
	}
\end{table}

\subsection{توسعه‌دهنده‌ها}
\begin{table}[H]
	\centering
	\resizebox{\textwidth}{!}{%
		\begin{tabular}{|C{0.2\textwidth}C{0.2\textwidth}C{0.2\textwidth}C{0.2\textwidth}C{0.2\textwidth}|}
			\hline
			\rowcolor[HTML]{F36619} 
			\multicolumn{5}{|c|}{\cellcolor[HTML]{fea82f}{\color[HTML]{FFFFFF}
					\Large
					\textbf{
						اعضا
					}
			}}                                   \\ \hline
			\rowcolor[HTML]{ffc288} 
			نام و نام خانوادگی
			&               
			مدرک تحصیلی    
			&                
			سابقه‌ی کاری
			&     
			شرح سوابق کاری
			&
			توانمندی‌ها
			\\
			\rowcolor[HTML]{fcecdd} 
			ایمان غلامی
			&               
			کارشناسی مهندسی کامپیوتر
			&    
			۳ سال
			&              
			تیم لیدر در مجموعه هزاردستان - بلد
			
			&
			\lr{Django, 
				Docker, Infrastructure, React}
			\\
			\rowcolor[HTML]{fcecdd} 
			حمیدرضا هدایتی
			&               
			کارشناسی مهندسی کامپیوتر
			&    
			۱ سال
			&              
			مدیر پروژه در مجموعه هزاردستان - ستون
			&
			\lr{
				Django, GoLang, Backend Development
			}
			\\
			\rowcolor[HTML]{fcecdd} 
			آرمین سعادت
			&               
			کارشناسی مهندسی کامپیوتر
			&    
			۵ ماه
			&              
			عضو فنی گروه آنالیسور
			&
			\lr{
				MongoDB,
				Node.js, 
				Database Design,
				Backend Development
			}
			\\
			\hline
		\end{tabular}
	}
\end{table}


\subsection{ارزیاب سامانه}
\begin{table}[H]
	\centering
	\resizebox{\textwidth}{!}{%
		\begin{tabular}{|C{0.25\textwidth}C{0.25\textwidth}C{0.25\textwidth}C{0.25\textwidth}|}
			\hline
			\rowcolor[HTML]{F36619} 
			\multicolumn{4}{|c|}{\cellcolor[HTML]{fea82f}{\color[HTML]{FFFFFF}
					\Large
					\textbf{
						ارزیاب سامانه
					}
			}}                                   \\ \hline
			\rowcolor[HTML]{ffc288} 
			نام و نام خانوادگی
			&               
			مدرک تحصیلی    
			&                
			سابقه‌ی کاری
			&     
			شرح سوابق کاری
			\\
			\rowcolor[HTML]{fcecdd} 
			حمیدرضا هدایتی
			&               
			کارشناسی مهندسی کامپیوتر
			&    
			۱ سال
			&              
					دبیر فنی رویداد 
					\lr{
					AI Challenge
					}
				و دبیر کل انجمن علمی دانشکده مهندسی کامپیوتر دانشگاه شریف
			\\
			\hline
		\end{tabular}
	}
\end{table}

\section{نکات درنظر گرفته شده در تشکیل تیم}
\subsection{حمیدرضا هدایتی}
به عنوان مدیر پروژه چندین محصول در شرکت ستون به خوبی از مدیریت این پروژه برمی‌آید. همچنین دانش و تجربه قوی در زمینه برنامه‌نویسی باعث می‌شود به خوبی از وظایف مربوط به تحلیل و پیاده‌سازی نیز برآید.
 \subsection{ایمان غلامی}
به عنوان رهبر فنی تیم در شرکت بلد به چالش‌های طراحی و پیاده‌سازی یک سیستم آگاه است. در زمینه زیرساخت دانش و تجربه بالایی دارد و به دلیل مهارت در جنگو در زمینه پیاده‌سازی و کدنویسی بک‌اند مهارت بالایی دارد.
\subsection{آرمین سعادت}
به دلیل تسلط بر پایتون و جنگو به خوبی از عهده وظایف تعیین شده برمی‌آید. به دلیل داشتن تجربه کاری به عنوان طراح پایگاه داده، پیاده‌سازی بک‌اند و فرانت‌اند می‌تواند تسک‌های مربوطه را با کیفیت مناسب به انجام رساند.

\section{آموزش‌های لازم}
برای طراحی و پیاده‌سازی بهتر پروژه اعضای تیم نیاز به گذراندن برخی آموزش‌ها، جهت کسب مهارت دارند که شرح آن‌ها در جدول 
زیر
قابل مشاهده‌است.


\begin{table}[H]
	\centering
	\label{training_table}
	\resizebox{\textwidth}{!}{%
		\begin{tabular}{|C{0.25\textwidth}C{0.75\textwidth}|}
			\hline
			\rowcolor[HTML]{F36619} 
			\multicolumn{2}{|c|}{\cellcolor[HTML]{fea82f}{\color[HTML]{FFFFFF}
					\Large
					\textbf{
						آموزش‌های لازم
					}
			}}                                   \\ \hline
			\rowcolor[HTML]{ffc288} 
			نام و نام خانوادگی
			&               
			مهارت‌های مورد نیاز
			\\
			\rowcolor[HTML]{fcecdd} 
			ایمان غلامی
			&               
				{\lr{UI/UX Design},
			\lr{CI/CD}
			 }
			\\
			\rowcolor[HTML]{fcecdd} 
			آرمین سعادت
			&           
{	\lr{UI/UX Design},
	\lr{CI/CD}
}
			\\
			\rowcolor[HTML]{fcecdd} 
			حمیدرضا هدایتی
			&         
{	\lr{UI/UX Design},
	\lr{CI/CD}
}
			\\
			\hline
		\end{tabular}
	}
\end{table}

\section{برنامه‌ی نشست‌ها}
برای هماهنگی بیشتر میان بخش‌های پروژه و بررسی روند پیشرفت کار، اعضا می‌بایستی دوشنبه هر هفته ساعت ۱۹ تا ۲۰ در جلسه شرکت‌کنند. البته این زمان تنها  زمان موجود نیست و در صورت نیاز به هماهنگی بیشتر می‌توان جلسات کوتاه دیگری را در روزهای دیگر هفته نیز برگزار کرد.
این جلسات با توجه به شرایط بیمای کرونا به صورت مجازی و آنلاین در پیام‌رسان اسکایپ برگزار می‌شود.  اعضای تیم می‌بایستی در این جلسات گزارشی از پیشرفت‌کار ارائه‌کنند و هماهنگی‌های لازم را با سایر اعضای گروه انجام دهند.
\section{دفعات و شیوه‌ی گزارش‌دهی}
\label{دفعات و شیوه‌ی گزارش‌دهی}
در جلسات روزهای 
دوشنبه
اعضا می‌بایستی گزارشی از پیشرفت کارهایشان ارائه دهند و بازخورد بگیرند.
مدیر پروژه با توجه به برنامه‌ریزی اولیه و با در نظر گرفتن پیشرفت کار، برنامه‌ی هفته آینده را تهیه کرده و پیش از شروع هفته آینده به دست اعضا می‌رساند.
هدایت این جلسات به عهده اسکرام مستر می‌باشد.

همچنین مدیر پروژه می‌بایستی پس از پایان هر فرسنگ‌نما
\LTRfootnote{milestone}
گزارشی کامل از جزئیات و پیشرفت روند کارهای پروژه به کارفرما ارائه‌کند.
دقت شود که در انتهای هر فرسنگ‌نما، سامانه موجود ارزیابی شده و نتیجه ارزیابی نیز در گزارشات ذکر می‌شود.

\section{مدیریت منازعه و بحران}
\subsection{مشارکت اعضا در جلسات}
\begin{enumerate}
	\item
	شرکت تمامی اعضای گروه در جلسات هفتگی الزامی‌است.
	\item 
	غیبت تنها با دلایل کاملا موجه و با اطلاع رسانی قبلی امکان‌پذیر است.
	\item
	در صورت غیبت، شخص غایب باید در جریان جلسه قرار گرفته و گزارشات خودش را به دست مدیر پروژه برساند. 
	\item
	در صورنی که بیش از ۲ نفر غیبت کنند جلسه لغو شده و یک روز دیگر در همان هفته به عنوان جلسه جبرانی برگزار خواهد شد.
	\item
	در صورت غیبت اسکرام مستر، رهبری جلسه به عهده مدیر پروژه خواهد بود.
	\item
غیبت‌ تحت هر شرایطی باید با اطلاع رسانی قبلی (تا ۲۴ ساعت قبل از جلسه) صورت گیرد.

	\item
	در صورت غیبت غیر مجاز یا بدون اطلاع‌رسانی قبلی، شخص خاطی نقدا جریمه خواهد شد.
\end{enumerate}

\subsection{منازعه میان اعضا}
در این شرایط اسکرام مستر، که رهبری جلسه را بر عهده دارد، با در نظر گرفتن رهیافت‌های اصلی موجود برای حل و فصل منازعات همچون:
\begin{itemize}
	\item
	تطبیق‌یافتن
	\LTRfootnote{Accommodating}
	\item 
	رقابت
	\LTRfootnote{Competing}
	\item 
	اجتناب
	\LTRfootnote{Avoiding}
		\item 
	همکاری
	\LTRfootnote{Collaborating}
		\item 
	مصالحه
	\LTRfootnote{Compromising}
\end{itemize}
می‌بایستی بهترین راه‌حل را در راستای رفع ناسازگاری و منازعه برگزیند.
\section{مدیریت گستره}
پس از جلسات هفتگی و با بررسی روند پیشرفت پروژه، مدیر پروژه گستره را با توجه به نمودار پرت با وضعیت فعلی مقایسه می‌کند. مدیر پروژه با همکاری اسکرام‌مستر، تسک‌های جدید را به اعضای گروه اطلاع می‌دهد. تعدادی از این تسک‌ها از پیش تعیین شده هستند که با توجه به عقب‌افتادگی‌های احتمالی دستخور تغییر شده‌اند. \newline
در حالت کلی، مدیر پروژه باید دقت داشته باشد که هر بخش از پروژه در زمان و بودجه تعریف شده قابل انجام است و ضمن رعایت کیفیت و کارایی مناسب به اتمام می‌رسد.
در شرایط نابسامان، مدیر پروژه می‌بایستی با تحلیل امکان‌سنجی، در صورت نیاز، بخش‌هایی از پروژه را با اولویت کمتر حذف کند تا منابع برای به اتمام رساندن پروژه مهیا شود.