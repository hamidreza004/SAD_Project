\chapter{رهیافت مدیریت}
\section{مدیریت، تجارب و وظایف}
\subsection{مدیر پروژه}
\begin{table}[H]
	\centering
	\resizebox{\textwidth}{!}{%
		\begin{tabular}{|C{0.2\textwidth}C{0.2\textwidth}C{0.2\textwidth}C{0.2\textwidth}C{0.2\textwidth}|}
			\hline
			\rowcolor[HTML]{F36619} 
			\multicolumn{5}{|c|}{\cellcolor[HTML]{008cff}{\color[HTML]{FFFFFF}
					\Large
					\textbf{
						مدیر پروژه
					}
			}}                                   \\ \hline
			\rowcolor[HTML]{c6d8f2} 
			نام و نام خانوادگی
			&               
			مدرک تحصیلی    
			&                
			سابقه‌ی کاری
			&     
			شرح سوابق کاری
			&
			توانمندی‌ها
			\\
			\rowcolor[HTML]{dee5ef} 
			امین رخشا
			&               
			کارشناسی مهندسی کامپیوتر
			&    
			دو ماه
			&              
				کارآموزی در شرکت رهنما،                   
				پروژه‌ی
				buyrapido
			&
			مدیریت پروژه
			\\
			\hline
		\end{tabular}
	}
\end{table}

\subsection{اعضا و سوابق و توانمندی‌ها}
تمامی اعضا در تمامی مراحل از جمله، تحلیل، توسعه و تست نقش خواهند داشت.
\begin{table}[H]
	\centering
	\resizebox{\textwidth}{!}{%
		\begin{tabular}{|C{0.2\textwidth}C{0.2\textwidth}C{0.2\textwidth}C{0.2\textwidth}C{0.2\textwidth}|}
			\hline
			\rowcolor[HTML]{F36619} 
			\multicolumn{5}{|c|}{\cellcolor[HTML]{008cff}{\color[HTML]{FFFFFF}
					\Large
					\textbf{
						اعضا
					}
			}}                                   \\ \hline
			\rowcolor[HTML]{c6d8f2} 
			نام و نام خانوادگی
			&               
			مدرک تحصیلی    
			&                
			سابقه‌ی کاری
			&     
			شرح سوابق کاری
			&
			توانمندی‌ها
			\\
		\rowcolor[HTML]{dee5ef} 
			امین رخشا
			&               
			کارشناسی مهندسی کامپیوتر
			&    
			دو ماه
			&              
			کارآموزی در شرکت رهنما،                   
			پروژه‌ی
			buyrapido
			&
			\lr{Node.js, 
				React Native, Android Programming, CSS, HTML}
			\\
			\rowcolor[HTML]{dee5ef} 
			مهبد مجید
			&               
			کارشناسی مهندسی کامپیوتر
			&    
			دو ماه
			&              
			کارآموزی در شرکت رهنما،                   
			پروژه‌ی
			buyrapido
			&
			\lr{
				MongoDB,
				Node.js, 
				Database Design,
				Backend Development
			}
			\\
			\rowcolor[HTML]{dee5ef} 
			کیمیا حمیدیه
			&               
			کارشناسی مهندسی کامپیوتر
			&    
			دو ماه
			&              
			تیم فنی
			
			\lr{AI Challenge}
			&
			\lr{Django, Docker, Infrastructure, CSS, HTML}
			\\
			\hline
		\end{tabular}
	}
\end{table}

\subsection{تحلیل‌گران}
\begin{table}[H]
	\centering
	\resizebox{\textwidth}{!}{%
		\begin{tabular}{|C{0.25\textwidth}C{0.25\textwidth}C{0.25\textwidth}C{0.25\textwidth}|}
			\hline
			\rowcolor[HTML]{F36619} 
			\multicolumn{4}{|c|}{\cellcolor[HTML]{008cff}{\color[HTML]{FFFFFF}
					\Large
					\textbf{
						تحلیل‌گران
					}
			}}                                   \\ \hline
			\rowcolor[HTML]{c6d8f2} 
			نام و نام خانوادگی
			&               
			مدرک تحصیلی    
			&                
			سابقه‌ی کاری
			&     
			شرح سوابق کاری
			\\
			\rowcolor[HTML]{dee5ef} 
			مهبد مجید
			&               
			کارشناسی مهندسی کامپیوتر
			&    
			دو ماه
			&              
			کارآموزی در شرکت رهنما،                   
			پروژه‌ی
			buyrapido
			\\
			\rowcolor[HTML]{dee5ef} 
			کیمیا حمیدیه
			&               
			کارشناسی مهندسی کامپیوتر
			&    
			دو ماه
			&              
			تیم فنی
			\lr{AI Challenge}
			\\
			\hline
		\end{tabular}
	}
\end{table}

\subsection{طراح پایگاه‌داده}
\begin{table}[H]
	\centering
	\resizebox{\textwidth}{!}{%
		\begin{tabular}{|C{0.25\textwidth}C{0.25\textwidth}C{0.25\textwidth}C{0.25\textwidth}|}
			\hline
			\rowcolor[HTML]{F36619} 
			\multicolumn{4}{|c|}{\cellcolor[HTML]{008cff}{\color[HTML]{FFFFFF}
					\Large
					\textbf{
					طراح پایگاه‌داده
					}
			}}                                   \\ \hline
			\rowcolor[HTML]{c6d8f2} 
			نام و نام خانوادگی
			&               
			مدرک تحصیلی    
			&                
			سابقه‌ی کاری
			&     
			شرح سوابق کاری
			\\
			\rowcolor[HTML]{dee5ef} 
			مهبد مجید
			&               
			کارشناسی مهندسی کامپیوتر
			&    
			دو ماه
			&              
			طراحی پایگاه‌داده و برنامه‌نویسی
			\lr{back-end}
			\\
			\hline
		\end{tabular}
	}
\end{table}


\subsection{طراحی گرافیکی و طراحی صفحات}
\begin{table}[H]
	\centering
	\resizebox{\textwidth}{!}{%
		\begin{tabular}{|C{0.25\textwidth}C{0.25\textwidth}C{0.25\textwidth}C{0.25\textwidth}|}
			\hline
			\rowcolor[HTML]{F36619} 
			\multicolumn{4}{|c|}{\cellcolor[HTML]{008cff}{\color[HTML]{FFFFFF}
					\Large
					\textbf{
						طراحی گرافیکی و طراحی صفحات
					}
			}}                                   \\ \hline
			\rowcolor[HTML]{c6d8f2} 
			نام و نام خانوادگی
			&               
			مدرک تحصیلی    
			&                
			سابقه‌ی کاری
			&     
			شرح سوابق کاری
			\\
			\rowcolor[HTML]{dee5ef} 
			امین رخشا
			&               
			کارشناسی مهندسی کامپیوتر
			&    
			چهار ماه
			&              
			طراحی صفحات وب و برنامه‌نویسی 
			\lr{front-end}
			\\
						\rowcolor[HTML]{dee5ef} 
			کیمیا حمیدیه
			&               
			کارشناسی مهندسی کامپیوتر
			&    
			دو ماه
			&              
			تیم فنی
			\lr{AI Challenge}
			و سابقه‌ی طراحی صفحات وب و کار با 
			\lr{Django}
			\\
			\hline
		\end{tabular}
	}
\end{table}

\subsection{توسعه‌دهنده‌ها}
\begin{table}[H]
	\centering
	\resizebox{\textwidth}{!}{%
		\begin{tabular}{|C{0.25\textwidth}C{0.25\textwidth}C{0.25\textwidth}C{0.25\textwidth}|}
			\hline
			\rowcolor[HTML]{F36619} 
			\multicolumn{4}{|c|}{\cellcolor[HTML]{008cff}{\color[HTML]{FFFFFF}
					\Large
					\textbf{
						توسعه‌دهنده‌ها
					}
			}}                                   \\ \hline
			\rowcolor[HTML]{c6d8f2} 
			نام و نام خانوادگی
			&               
			مدرک تحصیلی    
			&                
			سابقه‌ی کاری
			&     
			شرح سوابق کاری
			\\
			\rowcolor[HTML]{dee5ef} 
			امین رخشا
			&               
			کارشناسی مهندسی کامپیوتر
			&    
			دو ماه
			&              
			کارآموزی در شرکت رهنما،                   
			پروژه‌ی
			buyrapido
			\\
			\rowcolor[HTML]{dee5ef} 
			مهبد مجید
			&               
			کارشناسی مهندسی کامپیوتر
			&    
			دو ماه
			&              
			کارآموزی در شرکت رهنما،                   
			پروژه‌ی
			buyrapido
			\\
			\rowcolor[HTML]{dee5ef} 
			کیمیا حمیدیه
			&               
			کارشناسی مهندسی کامپیوتر
			&    
			دو ماه
			&              
			تیم فنی
			\lr{AI Challenge}
			\\
			\hline
		\end{tabular}
	}
\end{table}


\subsection{ارزیاب سامانه}
\begin{table}[H]
	\centering
	\resizebox{\textwidth}{!}{%
		\begin{tabular}{|C{0.25\textwidth}C{0.25\textwidth}C{0.25\textwidth}C{0.25\textwidth}|}
			\hline
			\rowcolor[HTML]{F36619} 
			\multicolumn{4}{|c|}{\cellcolor[HTML]{008cff}{\color[HTML]{FFFFFF}
					\Large
					\textbf{
						ارزیاب سامانه
					}
			}}                                   \\ \hline
			\rowcolor[HTML]{c6d8f2} 
			نام و نام خانوادگی
			&               
			مدرک تحصیلی    
			&                
			سابقه‌ی کاری
			&     
			شرح سوابق کاری
			\\
			\rowcolor[HTML]{dee5ef} 
			امین رخشا
			&               
			کارشناسی مهندسی کامپیوتر
			&    
			دو ماه
			&              
					سابقه‌ی ارزیابی پروژه‌های نرم‌افزاری
			\\
			\rowcolor[HTML]{dee5ef} 
			مهبد مجید
			&               
			کارشناسی مهندسی کامپیوتر
			&    
			یک ماه
			&              
	سابقه‌ی ارزیابی پروژه‌های نرم‌افزاری
			\\
			\rowcolor[HTML]{dee5ef} 
			کیمیا حمیدیه
			&               
			کارشناسی مهندسی کامپیوتر
			&    
			دو ماه
			&              
			سابقه‌ی ارزیابی پروژه‌های نرم‌افزاری
			\\
			\hline
		\end{tabular}
	}
\end{table}

\section{نکات درنظر گرفته شده در تشکیل تیم}
\subsection{امین رخشا}
با داشتن تجربه‌ی هدایت یک تیم در کارآموزی شرکت رهنما می‌تواند گروه را به خوبی مدیریت‌کند. همچنین با توجه به داشتن تجربه و تسلط به برنامه‌نویسی 
\lr{front-end}
 و وب می‌تواند به خوبی از عهده‌ی مسئولیت‌های مربوطه در بخش توسعه برآید.
 \subsection{مهبد مجید}
با داشتن تجربه‌ی طراحی سیستم‌های پایگاهی و توسعه‌ی
\lr{back-end}
در کارآموزی شرکت رهنما، می‌تواند به خوبی از عهده‌ی مسئولیت‌های مربوطه‌ برآید.
\subsection{کیمیا حمیدیه}
با داشتن تجربه‌ی عضویت در تیم فنی 
\lr{AI Challenge}
و تسلط بر زبان پایتون و 
\lr{Django}
می‌تواند به خوبی از عهده‌ی مسئولیت‌های مربوطه برآید.
\section{آموزش‌های لازم}
برای طراحی و پیاده‌سازی بهتر پروژه اعضای تیم نیاز به گذراندن برخی آموزش‌ها، جهت کسب مهارت دارند که شرح آن‌ها در جدول 
زیر
قابل مشاهده‌است.


\begin{table}[H]
	\centering
	\label{training_table}
	\resizebox{\textwidth}{!}{%
		\begin{tabular}{|C{0.25\textwidth}C{0.75\textwidth}|}
			\hline
			\rowcolor[HTML]{F36619} 
			\multicolumn{2}{|c|}{\cellcolor[HTML]{008cff}{\color[HTML]{FFFFFF}
					\Large
					\textbf{
						آموزش‌های لازم
					}
			}}                                   \\ \hline
			\rowcolor[HTML]{c6d8f2} 
			نام و نام خانوادگی
			&               
			مهارت‌های مورد نیاز
			\\
			\rowcolor[HTML]{dee5ef} 
			مهبد مجید
			&               
			\lr{MS Project, Django,
				\rl{زبان پایتون},
			\rl{معماری وب}
			 }
			\\
			\rowcolor[HTML]{dee5ef} 
			امین رخشا
			&           
			\lr{MS Project, Django,
				\rl{زبان پایتون},
				\rl{معماری وب}
			}    
			\\
			\rowcolor[HTML]{dee5ef} 
			کیمیا حمیدیه
			&         
			\lr{MS Project
			,\rl{معماری وب} }
			\\
			\hline
		\end{tabular}
	}
\end{table}
\section{برنامه‌ی نشست‌ها}
برای هماهنگی بیشتر میان بخش‌های پروژه و بررسی روند پیشرفت کار، اعضا می‌بایستی هر هفته، در روزهای 
چهارشنبه ساعت ۱۰ تا ۱۲
در جلسه شرکت‌کنند. البته این زمان تنها  زمان موجود نیست و در صورت نیاز به هماهنگی بیشتر می‌توان جلسات کوتاه دیگری را در روزهای دیگر هفته نیز برگزار کرد.
محل تشکیل این جلسات، سایت دانشکده‌ی کامپیوتر واقع در طبقه‌ی سوم است. در صورت تعطیلی سایت به هر دلیلی، جلسات در لابی دانشکده‌ی کامپیوتر برگزار می‌شوند. اعضای تیم می‌بایستی در این جلسات گزارشی از پیشرفت‌کار ارائه‌کنند و هماهنگی‌های لازم را با سایر اعضای گروه انجام دهند.
\section{دفعات و شیوه‌ی گزارش‌دهی}
\label{دفعات و شیوه‌ی گزارش‌دهی}
در جلسات روزهای 
چهارشنبه
اعضا می‌بایستی گزارشی از پیشرفت کارهایشان را به مدیر پروژه ارائه دهند و بازخورد بگیرند.
مدیر پروژه نیز با توجه به برنامه‌ریزی اولیه و پیشرفت کار اعضا، برنامه‌ای به روزشده  کرده و در اولین فرصت، پیش از شروع هفته‌ی آینده، به اعضای گروه ارسال می‌کند.

همچنین مدیر پروژه می‌بایستی پس از پایان هر فرسنگ‌نما
\LTRfootnote{milestone}
گزارشی کامل از جزئیات و پیشرفت روند کارهای پروژه به کارفرما ارائه‌کند.
\section{مدیریت منازعه و بحران}
\subsection{مشارکت اعضا در جلسات}
\begin{enumerate}
	\item
	شرکت تمامی اعضای گروه در جلسات هفتگی الزامی‌است.
	\item 
	یک جلسه غیبت در طی تمام جلسات بلامانع است.
	\item
	در صورت غیبت بیش از یک جلسه با صلاح‌دید مدیر، 
	جریمه‌ای در نظر گرفته خواهدشد.
\end{enumerate}

\subsection{منازعه میان اعضا}
در این شرایط مدیر گروه با در نظر گرفتن رهیافت‌های اصلی موجود برای حل و فصل منازعات همچون:
\begin{itemize}
	\item
	تطبیق‌یافتن
	\LTRfootnote{Accommodating}
	\item 
	رقابت
	\LTRfootnote{Competing}
	\item 
	اجتناب
	\LTRfootnote{Avoiding}
		\item 
	همکاری
	\LTRfootnote{Collaborating}
		\item 
	مصالحه
	\LTRfootnote{Compromising}
\end{itemize}
می‌بایستی بهترین راه‌حل را در راستای رفع ناسازگاری و منازعه برگزیند.
\section{مدیریت گستره}
همان‌طور که در بخش 
\ref{دفعات و شیوه‌ی گزارش‌دهی}
اشاره‌شد، پس از جلسات هفتگی و با بررسی روند پیش‌رفت پروژه، مدیر پروژه گستره را با توجه به نمودار پرت با وضعیت فعلی مقایسه می‌کند و با توجه به وضعیت برای جبران عقب‌ماندگی‌ها از برنامه، وظایف جدیدی را به اعضا تخصیص می‌دهد.

مدیر پروژه همچنین می‌بایستی با تحلیل امکان‌سنجی، در صورت نیاز، بخش‌هایی از پروژه را که اولویتی کمتر دارند را از پروژه حذف‌کند و گستره‌ی پروژه را به‌روزرسانی نماید. همچنین مدیر پروژه باید توجه داشته‌باشد که بودجه‌ی اختصاص‌داده‌شده به هر بخش پروژه، فراتر از حدود تعیین‌شده برای آن نرود.
