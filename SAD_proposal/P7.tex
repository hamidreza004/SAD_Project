	\chapter{برآوردها}
\section{برآورد زمانی}
برای برآورد زمانی ابتدا باید زمان کاری اعضای گروه را مشخص کنیم.

\subsection{زمان کاری}
با توجه به این که تمام اعضای گروه دانشجو هستند، زمان کاری را به صورت زیر است:

\begin{itemize}
	\item
	\textbf{روزهای شنبه تا چهارشنبه:}
	۲ ساعت، از ۷ تا ۹ شب
	\item 
	\textbf{روزهای پنجشنبه و جمعه:}
	۷ ساعت، ۱۰ تا ۱۲ صبح و ۲ تا ۷ بعد از ظهر
\end{itemize}

\subsection{بازدهی و وقفه‌ها}
مطمئناً هیچ‌کس همیشه در حین کار بازدهی کامل ندارد و همیشه وقفه‌هایی در کار رخ می‌دهد. این موارد باعث می‌شود که زمان انجام کارها بیشتر شود.  در تخمین‌ها ، بازدهی افراد ۷۵ درصد در نظر گرفته شده است و ۱۵ درصد کاهش سرعت هم به خاطر وقفه‌ها اضافه شده است.
\subsection{ساختار شکست کار
	\titlefootnote{Work Breakdown Structure}
}
برای این که بتوانیم تخمین دقیق‌تری از زمان داشته باشیم، باید پروژه را به بخش‌های کوچک‌تر و ملموس‌تر بشکانیم. این ساختار به صورت زیر است:
\begin{itemize}
	\item 
	پیشنهادنامه
	\begin{itemize}
		\item 
		انتخاب پروژه
		\item
		بررسی نمونه‌ها
		\item
		ساخت تمپلیت
		\item
		بخش اول
		\item
		بخش دوم
		\item
		بخش سوم
	\end{itemize}
	\item 
	تحلیل سامانه
	\begin{itemize}
		\item 
		تحلیل نیازمندی‌ها
		\item
		نمودار مورد کاربرد
		\begin{itemize}
			\item 
			پیدا کردن اکتورها
			\item 
			رسم نمودار
			\item 
			توضیحات
		\end{itemize}
		\item
		تعیین سناریوهای سیستم
		\item
		مستندسازی
	\end{itemize}
	\item 
	طراحی سامانه
	\begin{itemize}
		\item 
		نمودار داده رابطه‌ای
		\item 
		نمودار فعالیت و توالی
		\item 
		معماری سیستم
		\item 
		مستندسازی
		
	\end{itemize}
	\item
	پیاده‌سازی
	\begin{itemize}
		\item
		انتخاب تکنولوژی‌ها
		\item
		پیاده‌سازی پایگاه‌داده
		\item
		فاز یک پیاده‌سازی
		\begin{itemize}
			\item
			صفحه‌ی اصلی
			\item
			پروفایل
			\item
			ثبت‌نام و ورود
		\end{itemize}
		\item
		فاز دو پیاده‌سازی
		\begin{itemize}
			\item
			امکان اضافه کردن آگهی
			\item
			نمایش آگهی‌ها
		\end{itemize}
		\item
		فاز سه پیاده‌سازی
		\begin{itemize}
			\item
			امکانات جستجو و فیلترکردن
			\item
			سیستم پیشنهاددهنده
			\item
			ارائه‌ی آمار و بقیه‌ی امکانات
		\end{itemize}
		
		\item 
		تست و ارزیابی نهایی
		
	\end{itemize}
\end{itemize}

توجه داشته باشید فاز پیاده‌سازی، به روش چابک پیاده‌سازی می‌گردد. یعنی پس از مقداری پیاده‌سازی به بررسی و تست آن قسمت از سامانه می‌پردازیم.

پس برای هر یک از تسک‌های فاز سوم، تست وجود خواهد داشت. که برای سادگی در نمودار گانت نیامده است.



\subsection{روابط پیشنیازی}
از دیگر مواردی که بر زمان‌بندی تاثیر می‌گذارد، پیشنیازی در قالب نمودار پرت
\LTRfootnote{pert}
به پیوست ارسال شده است.

\subsection{برآورد زمانی انجام هر وظیفه}
تخمین زمانی انجام هر یک از وظایف ذکر شده به شرح زیر است:


\renewcommand{\arraystretch}{1.3}
\begin{table}[H]
	\centering
	\resizebox{\textwidth}{!}{%
		\begin{tabular}{|C{0.16666\textwidth}C{0.16666\textwidth}C{0.16666\textwidth}C{0.16666\textwidth}C{0.16666\textwidth}C{0.16666\textwidth}|}
			\hline
			\rowcolor[HTML]{F36619} 
			\multicolumn{6}{|c|}{\cellcolor[HTML]{008cff}{\color[HTML]{FFFFFF}
					\Large
					\textbf{
						برنامه‌ی زمانی
					}
			}}                                   \\ \hline
			\rowcolor[HTML]{c6d8f2} 
وظیفه & زمان خوش‌بینانه (ساعت)  & زمان واقع‌بینانه (ساعت) & زمان بدبینانه(ساعت) & میانگین & با احتساب وقفه و بازدهی
			\\
			\rowcolor[HTML]{dee5ef} 
			انتخاب پروژه &$ 0.5$ & 1 & $1.5$ & 1 & $1.6$ \\ 
			\rowcolor[HTML]{dee5ef} 
			بررسی نمونه‌ها & 1 & 2 & 3 & 2 & 3.1 \\ 
			\rowcolor[HTML]{dee5ef} 
			انتخاب نقش افراد & $0.5$& 1 & 2 & $1.1$ & $1.7$ \\ 
			\rowcolor[HTML]{dee5ef} 
			ساخت تمپلیت & 4 & 8 & 12 & 8 &$ 12.5$ \\ 
			\rowcolor[HTML]{dee5ef} 
			بخش اول & 2 & 5 & 8 & 5 &$ 7.8$ \\ 
			\rowcolor[HTML]{dee5ef} 
			بخش دوم & 4 & 7 & 10 & 7 & 11 \\ 
			\rowcolor[HTML]{dee5ef} 
			بخش سوم & 6 & 10 & 12 &
			 $9.7$
			  & $15.2$ \\ 
			\rowcolor[HTML]{dee5ef} 
			پایان پیشنهادنامه & 0 & 0 & 0 & 0 & 0 \\ 
			\rowcolor[HTML]{dee5ef} 
			تحلیل نیازمندی‌ها & 1 & 2 & 4 & 2.2 & $3.4$ \\ 
			\rowcolor[HTML]{dee5ef} 
			پیداکردن اکتورها & $0.1$ & $0.5$ & 1 & $0.5$ & $0.8$ \\ 
			\rowcolor[HTML]{dee5ef} 
			رسم نمودار & 1 & 3 & 4 & 2.8 & $4.4$ \\ 
			\rowcolor[HTML]{dee5ef} 
			توضیحات & 4 & 5 & 6 & 5 & $7.8$ \\ 
			\rowcolor[HTML]{dee5ef} 
			تعیین سناریوهای سیستم & 3 & 5 & 7 & 5 & $7.8$ \\ 
			\rowcolor[HTML]{dee5ef} 
			مستندسازی & 4 & 5 & 7 & $5.2$ & $8.1$ \\ 
			\rowcolor[HTML]{dee5ef} 
			پایان تحلیل & 0 & 0 & 0 & 0 & 0 \\ 
			\rowcolor[HTML]{dee5ef} 
			نمودار داده رابطه‌ای & 4 & 6 & 8 & 6 & $9.4$ \\ 
			\rowcolor[HTML]{dee5ef} 
			نمودار فعالیت و توالی & 5 & 7 & 10 & $7.2$ & $11.2$ \\ 
			\rowcolor[HTML]{dee5ef} 
			معماری سیستم & 4 & 6 & 10 & $6.3$ & $9.9$ \\ 
			\rowcolor[HTML]{dee5ef} 
			مستندسازی & 5 & 7 & 10 & $7.2$ & $11.2$ \\ 
			\rowcolor[HTML]{dee5ef} 
			پایان طراحی & 0 & 0 & 0 & 0 & 0 \\ 
			\rowcolor[HTML]{dee5ef} 
			انتخاب تکنولوژی‌ها & 1 & 2 & 3 & 2 & $3.1$ \\ 
			\rowcolor[HTML]{dee5ef} 
			پیاده‌سازی پایگاه‌داده & 10 & 15 & 25 & $15.8$ & $24.8$ \\ 
			\rowcolor[HTML]{dee5ef} 
			صفحه‌ی اصلی & 2 & 5 & 10 &$ 5.3$ & $8.4$ \\ 
			\rowcolor[HTML]{dee5ef} 
			پروفایل & 2 & 6 & 10 & 6 & $9.4$ \\ 
			\rowcolor[HTML]{dee5ef} 
			ثبت‌نام و ورود & 4 & 6 & 10 & $6.3$ & $9.9$ \\ 
			\rowcolor[HTML]{dee5ef} 
			امکان اضافه‌کردن آگهی & 7 & 9 & 11 & 9 & $14.1$ \\ 
			\rowcolor[HTML]{dee5ef} 
			نمایش آگهی‌ها & 7 & 10 & 12 & $9.8$ & $15.4$ \\ 
			\rowcolor[HTML]{dee5ef} 
			جست‌وجو و فیلترکردن & 6 & 8 & 12 & $8.3$ & $13.1$ \\ 
			\rowcolor[HTML]{dee5ef} 
			سیستم پیشنهاددهنده & 10 & 14 & 20 & $14.3$ & $22.5$ \\ 
			\rowcolor[HTML]{dee5ef} 
			آمار و بقیه‌ی امکانات & 6 & 10 & 12 &$ 9.7$ &$ 15.2$ \\ 
			\rowcolor[HTML]{dee5ef} 
			تست و ارزیابی نهایی & 3 & 4 & 6 &$ 4.2$ &$ 6.5$ \\ 
			\rowcolor[HTML]{c6d8f2} 
			\textbf{جمع} 
			& \textbf{$104.1$} & \textbf{$165.5$} & \textbf{$240.5$ }& \textbf{$167.8$} & \textbf{$269.7$} \\ 
			\hline
		\end{tabular}
	}
\end{table}

\renewcommand{\arraystretch}{1.7}

\subsection{برنامه‌ی زمانی}
با توجه به روابط پیشنیازی، تخمین زمان هر وظیفه و محدودیت‌هایی که در بخش ۶ ذکر شد، برنامه زمانی تنظیم شده است. این برنامه را در قالب نمودار گانت
\LTRfootnote{Gantt}
به پیوست ارسال شده است.

\section{برآورد مالی}
\subsection{دستمزدها}
دستمزد اعضای تیم به شرح زیر است:
\begin{itemize}
	\item 
	\textbf{امین رخشا}
	: ساعتی ۴۲ هزار تومان معادل ۳ دلار
	\item 
	\textbf{مهبد مجید}
	: ساعتی ۳۵ هزار تومان معادل 
	$2.5$
	دلار
	\item 
	\textbf{کیمیا حمیدیه}
	: ساعتی ۳۵ هزار تومان معادل 
	$2.5$
	دلار
\end{itemize}

\section{برآورد هزینه‌ها}
با توجه به دستمزد‌ها و تخمین‌های زمان انجام هر وظیفه، هزینه‌‌ی انجام بخش‌های مختلف پروژه به این صورت است:

\begin{table}[H]
	\centering
	\resizebox{\textwidth}{!}{%
		\begin{tabular}{|C{0.33\textwidth}C{0.33\textwidth}C{0.33\textwidth}|}
			\hline
			\rowcolor[HTML]{F36619} 
			\multicolumn{3}{|c|}{\cellcolor[HTML]{008cff}{\color[HTML]{FFFFFF}
					\Large
					\textbf{
						برآورد هزینه‌ی بخش‌ها
					}
			}}                                   \\ \hline
			\rowcolor[HTML]{c6d8f2} 
			\lr{ID}
			&               
			نام بخش
			&                
			هزینه
			\\
			\rowcolor[HTML]{dee5ef} 
			$1$
			&               
			شریف‌کار
			&    
			$752.66\$$
			\\
						\rowcolor[HTML]{dee5ef} 
			$2$
			&               
			پیشنهادنامه
			&    
			$140.93\$$
			\\
						\rowcolor[HTML]{dee5ef} 
			$11$
			&               
			تحلیل
			&    
			$86.13\$$
			\\
						\rowcolor[HTML]{dee5ef} 
			$20$
			&               
			طراحی
			&    
			$111.20\$$
			\\
									\rowcolor[HTML]{dee5ef} 
			$26$
			&               
			پیاده‌سازی
			&    
			$414.40\$$
			\\
			\hline
		\end{tabular}
		}
\end{table}