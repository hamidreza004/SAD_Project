\chapter{برآوردها}
برای اینکه روند انجام پروژه دقیق تر مشخص شود و چالش‌ها از ابتدا بررسی شده باشند تا اختلال زیادی در انجام پروژه به وجود نیاید لازم است از قسمت‌های مختلف پروژه برآورد زمانی و مالی داشته باشیم که جزئیات آن در ادامه آمده‌است.
\section{برآورد زمانی}
برای برآورد زمانی لازم است که ساعت کاری انجام پروژه، زمانی که هر یک از اعضا برای انجام پروژه دارد و درصد عملکرد هر نفر با در نظر گرفتن وقفه‌ها و میزان بازدهی مشخص شود. 
همچنین برای پروژه به قسمت‌های کوچک شکسته شود و زمان انجام هر قسمت توسط متخصص مربوطه پیش‌بینی شده و با در نظر گرفتن کار مفید اعضای تیم تخمینی از انجام کل پروژه به دست آورد. 

\subsection{زمان کاری}
تمامی اعضای تیم دانشجو هستند پس زمانی ککه می‌تواند کار‌ها در آن انجام شود به صورت زیر است (ساعت‌ها به صورت شناور در نظر گرفته شده است):

\begin{itemize}
	\item
	\textbf{روزهای شنبه تا چهارشنبه:}
	۲ ساعت کاری، از ۷ تا ۱۰ شب
	\item 
	\textbf{روزهای پنجشنبه و جمعه:}
	۸ ساعت کاری، از ۸ صبح تا ۶ بعد از ظهر 
\end{itemize}

\subsection{بازدهی و وقفه‌ها}
با توجه به اینکه بازدهی نمی‌تواند ۱۰۰ درصد باشد، بازدهی مورد انتظار را با توجه به نیرو‌های موجود ۷۵ درصد در نظر می‌گیریم. همچنین وقفه‌هایی مانند زمان ناهار که بازه‌ی ساعت کاری می‌افتند و سایر وقفه‌ها ۱۵ درصد زمان را نیز برای این وقفه‌ها در نظر می‌گیریم.
\subsection{ساختار شکست کار
	\titlefootnote{Work Breakdown Structure}
}
برای برآورد دقیق تر زمان انجام کار لازم است تا دید بهتری به انجام پروژه داشته باشیم، از این رو مراحل مختلف انچام پروژه در ادامه مشخص شده است:
\begin{enumerate}
	\item 
	پیشنهادنامه
	\begin{itemize}
		\item 
		انتخاب پروژه
		\item
		پیش نویس محصولی (بخش اول): مسئله، راه حل  و ذینفع‌ها
		\item
		پیش نویس انسانی (بخش دوم): منابع انسانی و فرآیند‌های انسانی
		\item
		پیش نویس فرآیندی (بخش سوم): امکان‌سنجی و فرآیند انجام پروژه
		\item
		انتخاب تمپلیت و نوشت نهایی
		\item
		ویرایش
		\item
		نمودار گنت\LTRfootnote{Gantt}
		\item
		نمودار پرت\LTRfootnote{Pert}
		\item
		ویرایش
	\end{itemize}
	\item 
	تحلیل سامانه
	\begin{itemize}
		\item 
		تحلیل نیازمندی‌ها
		\item
		نمودار مورد کاربرد
		\begin{itemize}
			\item 
			پیدا کردن اکتورها
			\item 
			رسم نمودار
			\item 
			توضیحات
		\end{itemize}
		\item
		تعیین سناریوهای سیستم
		\item
		مستندسازی
	\end{itemize}
	\item 
	طراحی سامانه
	\begin{itemize}
		\item 
		نمودار داده رابطه‌ای
		\item 
		نمودار فعالیت و توالی
		\item
		طراحی تجربه کاربری
		\item
		 طراحی رابط کاربری
		 \item 
		معماری سیستم
		\item 
		مستندسازی
		
	\end{itemize}
	\item
	پیاده‌سازی
	\begin{itemize}
		\item
		انتخاب تکنولوژی‌ها
		\item
		پیاده‌سازی پایگاه‌داده
		\item
		فاز یک پیاده‌سازی
		\begin{itemize}
			\item
			صفحه‌ی اصلی
			\item
			ثبت‌نام، ورود و خروج
			\item
			داشبورد
		\end{itemize}
		\item
		فاز دو پیاده‌سازی
		\begin{itemize}
		    \item
		    دعوت و ایجاد مخاطب
		    \item
		    قابلیت ایجاد بدهی و طلب
			\item
			نمایش آمار بدهی‌ها و طلب‌ها
		\end{itemize}
		\item
		فاز سه پیاده‌سازی
		\begin{itemize}
			\item
			قابلیت ایجاد گروه
			\item
			تقسیم بندی‌های مختلف در گروه
			\item
			ارائه‌ی آمار و بقیه‌ی امکانات
		\end{itemize}
		
		\item 
		تست و ارزیابی نهایی
		
	\end{itemize}
\end{enumerate}

با توجه به اینکه فرآیند انجام پروژه به صورت چابک است. مراحل تست و ارزیابی بعد از هر پپیاده سازی انجام می‌شود.


\subsection{روابط پیشنیازی}

با توجه به اینکه برای انجام برخی از وظایف لازم است تا وظایف دیگری انجام شود، لازم است تا روابط پیش‌نیازی را داشته باشیم.  نمودار پرت\LTRfootnote{pert}
به پیوست ارسال شده است و این موارد را مشخص می‌کند.

\subsection{برآورد زمانی انجام هر وظیفه}
تخمین زمانی انجام هر یک از وظایف ذکر شده به شرح زیر است:


\renewcommand{\arraystretch}{1.1}
\begin{table}[H]
	\centering
	\resizebox{\textwidth}{!}{%
		\begin{tabular}{|C{0.16666\textwidth}C{0.16666\textwidth}C{0.16666\textwidth}C{0.16666\textwidth}C{0.16666\textwidth}C{0.16666\textwidth}|}
			\hline
			\rowcolor[HTML]{F36619} 
			\multicolumn{6}{|c|}{\cellcolor[HTML]{008cff}{\color[HTML]{FFFFFF}
					\Large
					\textbf{
						برنامه‌ی زمانی
					}
			}}                                   \\ \hline
			\rowcolor[HTML]{c6d8f2} 
وظیفه & زمان خوش‌بینانه (ساعت)  & زمان واقع‌بینانه (ساعت) & زمان بدبینانه(ساعت) & میانگین & با احتساب وقفه و بازدهی
			\\

\rowcolor[HTML]{dee5ef}
پیشنهاد نامه & $13$ & $23$ & $34.5$ & $23.5$ & $39.2$ \\
\rowcolor[HTML]{dee5ef}
انتخاب پروژه & $0.5$ & $1$ & $2$ & $1.2$ & $2$ \\
\rowcolor[HTML]{dee5ef}
پیش نویس محصولی & $2$ & $3$ & $4$ & $3$ & $5$ \\
\rowcolor[HTML]{dee5ef}
پیش نویس انسانی & $1$ & $2$ & $3$ & $2$ & $3.3$ \\
\rowcolor[HTML]{dee5ef}
پیش نویس فرآیندی & $3$ & $4$ & $5$ & $4$ & $6.7$ \\
\rowcolor[HTML]{dee5ef}
انتخاب تمپلیت و نوشت نهایی & $3$ & $5$ & $8$ & $5.3$ & $8.8$ \\
\rowcolor[HTML]{dee5ef}
ویرایش & $0.5$ & $1$ & $1.5$ & $1$ & $1.7$ \\
\rowcolor[HTML]{dee5ef}
نمودار Gantt & $2$ & $4$ & $6$ & $4$ & $6.7$ \\
\rowcolor[HTML]{dee5ef}
نمودار Pert & $1$ & $3$ & $5$ & $3$ & $5$ \\
\rowcolor[HTML]{dee5ef}
تحلیل سامانه & $10.5$ & $24$ & $33$ & $22.6$ & $37.7$ \\
\rowcolor[HTML]{dee5ef}
تحلیل نیازمندی‌ها & $1$ & $2$ & $3.5$ & $2.2$ & $3.7$ \\
\rowcolor[HTML]{dee5ef}
تعیین سناریو‌های سیستم & $2$ & $5$ & $7$ & $4.7$ & $7.8$ \\
\rowcolor[HTML]{dee5ef}
نمودار کاربرد & $2.5$ & $8$ & $10.5$ & $7$ & $11.7$ \\
\rowcolor[HTML]{dee5ef}
پیدا کردن اکتور‌ها & $0.5$ & $1$ & $1.5$ & $1$ & $1.7$ \\
\rowcolor[HTML]{dee5ef}
رسم نمودار & $1$ & $3$ & $4$ & $2.7$ & $4.5$ \\
\rowcolor[HTML]{dee5ef}
توضیحات & $1$ & $4$ & $5$ & $3.3$ & $5.5$ \\
\rowcolor[HTML]{dee5ef}
مستند سازی  & $3$ & $4$ & $5$ & $4$ & $6.7$ \\
\rowcolor[HTML]{dee5ef}
طراحی سامانه & $23$ & $38$ & $46$ & $35.8$ & $59.6$ \\
\rowcolor[HTML]{dee5ef}
نمودار‌ داده رابطه‌ای & $6$ & $8$ & $9$ & $7.7$ & $12.8$ \\
\rowcolor[HTML]{dee5ef}
نمودار فعالیت و توالی & $5$ & $10$ & $11$ & $8.7$ & $14.5$ \\
\rowcolor[HTML]{dee5ef}
معماری سیستم & $3$ & $6$ & $9$ & $6$ & $10$ \\
\rowcolor[HTML]{dee5ef}
طراحی تجربه کاربری & $3$ & $5$ & $6$ & $4.7$ & $7.8$ \\
\rowcolor[HTML]{dee5ef}
طراحی رابطه کاربری & $3$ & $5$ & $6$ & $4.7$ & $7.8$ \\
\rowcolor[HTML]{dee5ef}
مستند سازی  & $3$ & $4$ & $5$ & $4$ & $6.7$ \\
\rowcolor[HTML]{dee5ef}
پیاده سازی & $105$ & $140$ & $181$ & $142.1$ & $236.7$ \\
\rowcolor[HTML]{dee5ef}
انتخاب تکنولوژی & $3$ & $9$ & $12$ & $8$ & $13.3$ \\
\rowcolor[HTML]{dee5ef}
پیاده سازی پایگاه داده & $8$ & $10$ & $14$ & $10.7$ & $17.8$ \\
\rowcolor[HTML]{dee5ef}
فاز یک پیاده‌سازی & $28$ & $35$ & $42$ & $35$ & $58.3$ \\
\rowcolor[HTML]{dee5ef}
صفحه اصلی & $9$ & $12$ & $15$ & $12$ & $20$ \\
\rowcolor[HTML]{dee5ef}
ثبت‌نام، ورود و خروج & $13$ & $15$ & $18$ & $15.3$ & $25.5$ \\
\rowcolor[HTML]{dee5ef}
داشبورد & $6$ & $8$ & $9$ & $7.7$ & $12.8$ \\
\rowcolor[HTML]{dee5ef}
فاز دو پیاده‌سازی & $29$ & $42$ & $55$ & $42$ & $70$ \\
\rowcolor[HTML]{dee5ef}
دعوت و ایجاد مخاطب & $14$ & $18$ & $24$ & $18.7$ & $31.2$ \\
\rowcolor[HTML]{dee5ef}
قابلیت ایجاد بدهی و طلب & $9$ & $12$ & $15$ & $12$ & $20$ \\
\rowcolor[HTML]{dee5ef}
نمایش آمار بدهی‌ها و طلب‌ها & $6$ & $12$ & $16$ & $11.3$ & $18.8$ \\
\rowcolor[HTML]{dee5ef}
فاز سه پیاده سازی & $34$ & $39$ & $51$ & $41.4$ & $69$ \\
\rowcolor[HTML]{dee5ef}
ایجاد گروه & $13$ & $15$ & $19$ & $15.7$ & $26.2$ \\
\rowcolor[HTML]{dee5ef}
تقسیم بندی‌های مختلف در گروه & $6$ & $8$ & $10$ & $8$ & $13.3$ \\
\rowcolor[HTML]{dee5ef}
ارائه‌ی آمار به ادمین و بقیه امکانات & $15$ & $16$ & $22$ & $17.7$ & $29.5$ \\
\rowcolor[HTML]{dee5ef}
تست و ارزیابی نهایی & $3$ & $5$ & $7$ & $5$ & $8.3$ \\
\rowcolor[HTML]{dee5ef}
جمع & $151.5$ & $225$ & $294.5$ & $224$ & $373.2$ \\


			\hline
		\end{tabular}
	}
\end{table}

\renewcommand{\arraystretch}{1.7}

\subsection{برنامه‌ی زمانی}
با توجه به موارد تحلیل شده در قسمت‌های قبل تصویر کلی از برآورد زمان اجرا با در نظر گرفتن انجام کار‌ها به صورت موازی و حفظ پیش‌نیازی، در نمودار گانت
\LTRfootnote{Gantt}
به پیوست ارسال شده است.

\section{برآورد مالی}
\subsection{دستمزدها}
دستمزد اعضای تیم به شرح زیر است:
\begin{itemize}
	\item 
	\textbf{آرمین سعادت}
	: ساعتی ۱۵۰ هزار تومان معادل ۶ دلار
	\item 
	\textbf{ایمان غلامی}
	: ساعتی ۱۳۸ هزار تومان معادل 
	$5.5$
	دلار
	\item 
	\textbf{حمیدرضا هدایتی}
	: ساعتی ۱۳۸ هزار تومان معادل 
	$5.5$
	دلار
\end{itemize}

\subsection{هزینه خدمات مورد استفاده}
خدماتی که حین انجام پروژه لازم است تا مورد استفاده قرار گیرد به شرح زیر است:
\begin{itemize}
	\item 
	\textbf{سرور}
	: ماهیانه ۳۰۰ هزار تومان معادل ۱۲ دلار
	\item 
	\textbf{پنل پیامک}
	:
	هزینه اولیه پنل ۵۰۰ هزار تومان معادل ۲۰ دلار
	\\
	به ازای هر ۱۰۰۰ پیامک ۲۵ هزار تومان معادل 
	$2$
	دلار

\end{itemize}

\subsection{برآورد هزینه‌ها}
با توجه به دستمزد‌ها و تخمین زمانی و نمودار گانت مجموع هزینه‌ها به شرح زیر است:

\begin{table}[H]
	\centering
	\resizebox{\textwidth}{!}{%
		\begin{tabular}{|C{0.33\textwidth}C{0.33\textwidth}C{0.33\textwidth}|}
			\hline
			\rowcolor[HTML]{F36619} 
			\multicolumn{3}{|c|}{\cellcolor[HTML]{008cff}{\color[HTML]{FFFFFF}
					\Large
					\textbf{
						برآورد هزینه‌ی بخش‌ها
					}
			}}                                   \\ \hline
			\rowcolor[HTML]{c6d8f2} 
			\lr{ID}
			&               
			نام بخش
			&                
			هزینه
			\\
			\rowcolor[HTML]{dee5ef} 
			$1$
			&               
			انجام کامل پروژه
			&    
			$752.66\$$
			\\
						\rowcolor[HTML]{dee5ef} 
			$2$
			&               
			پیشنهادنامه
			&    
			$140.93\$$
			\\
						\rowcolor[HTML]{dee5ef} 
			$11$
			&               
			تحلیل
			&    
			$86.13\$$
			\\
						\rowcolor[HTML]{dee5ef} 
			$20$
			&               
			طراحی
			&    
			$111.20\$$
			\\
									\rowcolor[HTML]{dee5ef} 
			$26$
			&               
			پیاده‌سازی
			&    
			$414.40\$$
			\\
			\hline
		\end{tabular}
		}
\end{table}
