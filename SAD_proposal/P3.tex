
\makeatletter
\def\titlefootnote{\ifx\protect\@typeset@protect\expandafter\LTRfootnote\else\expandafter\@gobble\fi}
\makeatother

\chapter{گستره}
	\section{ذینفع‌های سامانه \titlefootnote{Stakeholders}
	}
	\subsection{مالکین}
	مالک این سامانه دکتر حیدرنوری استاد درس تحلیل و طراحی سیستم‌ها است. انتظار ایشان از این سامانه عملکرد مناسب در حد به کار گرفته شدن در محیط واقعی است.
	
	\subsection{کاربران}
	کاربران این سامانه سه دسته هستند:
\begin{enumerate}
\item \textbf{کارفرمایان:}
	کارفرمایان به دنبال استخدام یک فرد برای سازمان خود هستند. این کاربران برای موقعیت شغلی سازمان خود یک آگهی در سامانه درج می‌کنند. آن‌ها انتظار دارند که این آگهی به سرعت به تمام کارجویانی که مهارت‌های مورد نیاز این موقعیت را دارند نمایش داده شود تا بتوانند سریعتر یک فرد مناسب پیدا کنند.
\item \textbf{کارجویان:}
 کارجویان دانشجویان و فارغ‌التحصیلان دانشگاه شریف هستند. آن‌ها در این سامانه از بین موقعیت‌های شغلی که کارفرمایان ارائه کرده‌اند برای خود به دنبال کار می‌گردند.
\item \textbf{مدیران سامانه:}
 تعدادی از کاربران در سامانه نقش مدیریت ‌آن‌ را دارند و با کارهایی از قبیل بستن حساب‌های کاربری در صورت هر گونه تخلف به عملکرد سیستم کمک می‌کنند. این کاربران می‌توانند لیست حساب‌های کاربری و حساب کاربری هر فرد هم ببینند.
\end{enumerate}

\subsection{تحلیلگران، طراح‌ها و سازندگان سامانه}

تیم ۳ نفره‌ی معرفی شده در آغاز پیشنهادنامه به شکل گروهی هر سه وظیفه‌ی تحلیل، طراحی و توسعه‌ی سامانه را بر عهده دارد.

\subsection{مدیر پروژه}

\subsection{فراهم کنندگان سرویس خارجی}
از 
\lr{Google Maps} 
برای نشان دادن موقعیت جغرافیایی موقعییت‌های شغلی در نقشه استفاده می‌شود.

\section{داده‌ها}
داده‌هایی که توسط این سیستم مدیریت می‌شوند موارد زیر هستند:
\begin{enumerate}
\item \textbf{اطلاعات کاربران:}
هر کاربر این سامانه یک پروفایل دارد. این پروفایل برای کارجویان شامل مواردی مانند رزومه، علاقه‌مندی‌ها (برای دریافت آگهی‌های شغلی مرتبط در حساب کاربری شخصی)، مهارت‌ها و غیره است. برای کارفرمایان هم این پروفایل مواردی مانند اطلاعات مربوط به کسب و کار کارفرما را در بر دارد. همچنین اطلاعات ارتباط میان این دو دسته کاربر مانند امتیازهایی که هر کارجو به  هر کارفرما و برعکس داده است هم ذخیره می‌شود.

\item \textbf{آگهی‌ها:}
آگهی‌هایی که کارفرمایان در سیستم ثبت کرده‌اند با مشخصات آن‌ها از قبیل نوع کار (کارآموزی، دائمی یا پاره‌وقت)، لیست پیشنیازهای فرد ذیصلاح برای شغل، مهارت‌های مورد انتظار، سابقه‌ی ‌کاری، حقوق و مزایا، ساعت کاری، محل شرکت (آدرس کامل به همراه موقعیت جغرافیای بر روی نقشه)، جنسیت، شرایط کاری و غیره نگه‌داری می‌شوند.
\item \textbf{تراکنش‌ها:}
تمام تراکنش‌ها و اتفاقات در سامانه از قبیل جستجوها، درج‌های آگهی، بازدید‌ها از آگهی‌ها و پروفایل‌ها، تراکنش‌های کاربران به همراه تاریخ برای ارائه‌ی آمار نگه‌داری می‌شوند.
\end{enumerate}

\section{امکانات}
\subsection{
امکانات مربوط به کارفرمایان}
\begin{itemize}
	\item
	امکان درج آگهی شغلی توسط کارفرما با توصیفات آن
	\item
	امکان جستجوی دانشجویان/فارغ التحصیلان بر اساس مهارتها، موقعیت مکانی، سابقه و ... 
	\item
	امکان مشاهده پرونده دانشجویان/فارغ التحصیلان پیشنهادی
	\item
	امکان استخدام دانشجویان/فارغ التحصیلان
	 \item
	پیشنهاد مرتبط‌‌‌ ‌ترین افراد بر اساس مهارت‌های آن‌ها به کارفرمایان بر اساس نیازمندی‌های آنها
\end{itemize}
\subsection{
امکانات مربوط دانشجویان و فارغ‌التحصیلان}
\begin{itemize}
	\item 
امکان ایجاد حساب کاربری (دریافت اطلاعات مورد نیاز از هر فرد) و تایید ثبت‌نام 
\item 
امکان تعریف مهارت‌های فردی، بارگذاری رزومه و ویرایش حساب کاربری 
 \item 
امکان جستجوی کارفرما بر اساس مهارت‌های مورد نیاز، موقعیت مکانی، سابقه (امتیاز) و ...  
\item
امکان جستجوی کار بر اساس مهارت‌های مورد نیاز، موقعیت مکانی، حقوق و ...  
 \item 
امکان نظر و امتیازدهی به کارفرما
   \item 
پیشنهاد آگهی های مرتبط با علاقه‌مندی دانشجویان/فارغ التحصیلان به آنها
\end{itemize}
\subsection{
امکانات عمومی}
\begin{itemize}
	\item
	نمایش اطلاعات آماری نظیر کاربرانی که بیشتر مورد جستجو بوده‌اند، پرکارترین کارفرمایان،
استخدامی‌های اخیر، و آمار کلی مراجعه به سامانه در صفحه اصلی سامانه 
	\item
	نمایش تمامی تراکنش‌های انجام شده در پروفایل کاربر بر اساس تاریخ
	\item
	دسترسی متفاوت به سامانه توسط مدیر سامانه و امکان مشاهده لیست کاربران سامانه و حساب کاربری هر فرد و امکان بستن حساب هر فرد به علت هرگونه تخلف
\end{itemize}

\section{گستردگی مکانی}
مکان موقعیت‌های شغلی در این سامانه با توجه به این که فارغ‌التحصیلان شریف می‌توانند در هر جایی باشند و همچنین دانشجویان برای کارآموزی می‌توانند به خارج از کشور هم بروند محدودیتی ندارد و هر جایی می‌تواند باشد.
