
\makeatletter
\def\titlefootnote{\ifx\protect\@typeset@protect\expandafter\LTRfootnote\else\expandafter\@gobble\fi}
\makeatother

\chapter{گستره}
	\section{ذینفع‌های سامانه \titlefootnote{Stakeholders}
	}
	\subsection{مالکین}
	مالک این سامانه دکتر حیدرنوری استاد درس تحلیل و طراحی سیستم‌ها است. انتظار ایشان از این سامانه عملیاتی شدن سامانه برای ارائه به کاربران در صورت وجود زیرساخت‌های مناسب و برآورده شدن حداقل‌های مالی است.
	
	\subsection{کاربران}
کاربران این سامانه دو دسته هستند:
\begin{enumerate}
\item \textbf{کاربران عمومی:}
	کاربرانی هستند که با ثبت‌نام در سامانه و تکمیل کردن حساب کاربری خود به دنبال اضافه کردن دوست در  سامانه و ایجاد گروه‌ها هستند که خرج‌ها را وارد کرده و بدهی‌ها را پرداخت کنند.

\item \textbf{مدیران سامانه:}
مدیران سامانه با دسترسی به اطلاعات تمامی کاربران و گروه‌ها توانایی مشاهده تمامی گزارشات مالی سامانه را دارند.
\end{enumerate}

\subsection{تحلیلگران، طراح‌ها و سازندگان سامانه}

تیم ۳ نفره‌ی معرفی شده در آغاز پیشنهادنامه به شکل گروهی هر سه وظیفه‌ی تحلیل، طراحی و توسعه‌ی سامانه را بر عهده دارد.

\subsection{مدیر پروژه}

\subsection{فراهم کنندگان زیرساخت‌های سرویس و سرویس‌های خارجی}
از 
\lr{Google Maps} 
برای نشان دادن موقعیت جغرافیایی موقعییت‌های شغلی در نقشه استفاده می‌شود همچنین از زیرساخت ابری \lr{ArvanCloud} برای نگه‌داری و ذخیره‌سازی اطلاعات بر روی فضای ابری استفاده خواهد شد، همچنین برای سرویس \lr{CI/CD} از سرویس \lr{Github} استفاده می‌شود.

\section{داده‌ها}
داده‌هایی که توسط این سیستم مدیریت می‌شوند موارد زیر هستند:
\begin{enumerate}
\item \textbf{اطلاعات کاربران:}
هر کاربر این سامانه یک حساب کاربری دارد. این حساب کاربری شامل ایمیل و شماره تلفن می‌شود که هنگام ثبت‌نام کاربر از او گرفته می‌شود. هر کاربر علاوه بر اطلاعات شخصی خود تعدادی دوست در سامانه دارد که می‌تواند با آن‌ها خرج رد و بدل کند. هر کاربر علاوه بر حساب کاربری میزان بدهی کلی و میزان طلب کلی‌ا‌ش مشخص شده است.
\item \textbf{گروه‌ها:}
گروه‌ها تعدادی از کاربران هستند که توسط یک نفر تشکیل می‌شوند. به ازای هر گروه اطلاعاتی مانند خرج‌های آن گروه ذخیره شده است.
\item \textbf{خرج‌ها:}
هر خرج یک کاربر به عنوان خرج کننده و یک کاربر یا گروه به عنوان بدهکار دارد، هر خرج اطلاعاتی مانند عکس، موقعیت جغرافیایی، تاریخ و اسم می‌تواند داشته باشد. خرج‌های هر کاربر قابل دسترسی هستند.
\end{enumerate}

\section{امکانات}
\subsection{
امکانات مربوط به کاربران} 
\begin{itemize}
	\item
	امکان ثبت خرج برای تک نفر
	\item
	امکان ثبت خرج برای یک گروه به صورت مساوی
	\item
	امکان ثبت خرج برای یک گروه به صورت وزن‌دار
	\item
	امکان مشاهده تاریخه بدهی‌ها و طلب‌ها
	 \item
	امکان اضافه کردن یک دوست 
		 \item
	امکان ساخت گروه با دوستان
		 \item
	امکان پرداخت یک بدهی 
\end{itemize}
\subsection{
امکانات مربوط به مدیران سامانه}
\begin{itemize}
	\item 
امکان ایجاد حساب کاربری (دریافت اطلاعات مورد نیاز از هر فرد) و تایید ثبت‌نام 
\item 
امکان ویرایش حساب‌های کاربری، لیست دوستان هر فرد و افراد گروه‌ها
 \item 
امکان ویرایش خرج‌های فردی و گروهی
\item
امکان مشاهده گزارش‌های مالی هر کاربر
\end{itemize}

\section{گستردگی مکانی}
با توجه به اینکه نسخه اولیه قابل ارائه سرویس به زبان فارسی ارائه می‌شود تمامی فارسی زبانان می‌توانند از این سرویس استفاده کنند.
