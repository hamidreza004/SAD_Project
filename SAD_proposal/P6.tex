\chapter{محدودیت‌ها}
\section{زمان شروع}
برای شروع بخش‌های مختلف باید آن مبحث تدریس شود و توضیحات آن به دست تیم برسد. با فرض این که بعد از هر سررسید این شرایط برای بخش بعد برقرار است برنامه‌ریزی می‌کنیم.
\section{سررسیدها}
۵ سررسید در این پروژه داریم که به شرح زیر است:
\begin{enumerate}
	\item 
	تحویل پیشنهادنامه : شنبه ۱۹ آبان
	\item 
	تحویل نمودارهای مورد کاربرد: شنبه ۲۶ آبان
	\item 
	تحویل نمودار داده رابطه‌ای: شنبه ۱۰ آذر
	\item
	تحویل نمودارهای فعالیت و توالی: شنبه ۲۴ آذر
	\item
	معماری سامانه: شنبه ۸ دی
	\item
	پایان پیاده‌سازی: جمعه  ۲۸ دی
\end{enumerate}

\section{بودجه}
بودجه زمانی: طبق نمودار گانت به ۵۵ روز برای انجام کار نیاز داریم.\\
بودجه مالی: سقف بودجه‌ی این پروژه ۲۰ میلیون تومان در نظر گرفته شده است. این مبلغ در ۳ مرحله به صورت زیر پرداخت می‌شود:
\begin{enumerate}
	\item 
	۲۰ درصد به صورت پیش‌پرداخت
	\item 
	۳۰ درصد بعد از فاز دوم پیاده‌سازی
	\item 
	۵۰ درصد بعد از تحویل نهایی
\end{enumerate}

\section{تکنولوژی}
این پروژه را در بستر وب پیاده می‌شود. دلایل این انتخاب به این شرح است:
\begin{enumerate}
	\item
	با توجه به این که موقعیت مکانی افراد پراکنده است، تنها دو گزینه‌‌ی وب و اپلیکیشن مطرح است.
	\item 
	با توجه به این که مهلت پیاده سازی محدود است، باید یک بستر برای توسعه استفاده کنیم. مشکل گزینه‌ی دیگر یعنی اپلیکیشن این است که اگر برای هر تمام سیستم‌عامل ‌های پر استفاده توسعه داده نشود، بخش زیادی از کاربران نمی‌توانند از آن استفاده کنند.
	\item
	اعضای تیم به پیاده‌سازی در این بستر مسلط‌تر هستند.
	\item 
	سرعت توسعه برای زمان در دسترس مناسب است.
\end{enumerate}
