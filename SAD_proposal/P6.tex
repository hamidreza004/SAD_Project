\chapter{محدودیت‌ها}
محدوودیت‌های موجود برای انجام این پروژه وجود دارد که لازم است در روند انجام پروژه در نظر گرفته شود. این محدودیت‌ها از شامل زمان، بودجه و تکنولوژی است.
\section{زمان}
\subsection{زمان شروع}
از آنجایی که شروع فاز‌های مختلف نیاز‌مند یادگیری آن بخش است، برای زمان شروع هر قسمت محدودیت وجود دارد.
\subsection{سررسیدها}
همچنین در هر بخش سررسید‌هایی
(\lr{Deadline})
مشخص می‌شود که لازم است بخش‌هایی از پروژه تا سررسید‌ها انجام شود. 
۵ سررسید در این پروژه داریم که به شرح زیر است:
\begin{enumerate}
	\item 
	تحویل پیشنهادنامه : شنبه ۴ اردیبهشت
	\item 
	تحویل نمودارهای مورد کاربرد: شنبه ۲۵ اردیبهشت
	\item 
	تحویل نمودار داده رابطه‌ای: شنبه ۸ خرداد
	\item
	تحویل نمودارهای فعالیت و توالی: شنبه ۲۲ خرداد
	\item
	معماری سامانه: شنبه ۲۹ خرداد
	\item
	پایان پیاده‌سازی: شنبه  ۲ مرداد
\end{enumerate}

\section{بودجه}
بودجه‌ای که برای این پروژه در نظر گرفته شده است ۱۲۰ میلیون تومان است که در دو مرحله پرداخت می‌شود.
\begin{enumerate}
	\item 
	۷۵ درصد به صورت پیش‌پرداخت به مبلغ ۹۰ میلیون تومان
	\item 
	۲۵ درصد بعد از تحویل نهایی به مبلغ ۳۰ میلیون تومان
\end{enumerate}

\section{تکنولوژی}
این پروژه به صورت وب‌اپلیکشن پیاده می‌شود که کاربر با مراجعه به یک آدرس اینترنتی می‌تواند از این سرویس استفاده کند.\\
با توجه به کاربران این سیستم و همچنین تمایل جامعه کاربران گزینه‌های موجود اپلیکشن‌ و وب‌اپلیکیشن‌ است.
دلایل انتخاب وب‌اپلیکیشن به شرح زیر است.
\begin{enumerate}
    \item
    وب‌اپلیکیشن‌ها وابسته به سیستم عامل نیستند و جامعه گسترده‌تری را پوشش 
    می‌دهد.
    \item
    از طرفی استقلال وب‌اپلیکشن‌ها از سیستم عامل باعث می‌شود که هزینه پیاده‌سازی برای سیستم عامل‌های مختلف پرداخت نشود که باعث صرفه‌جویی در بودجه، زمان و استفاده از تخصص‌های مختلف است.
	\item
	اگر طراحی درستی در پیاده‌سازی وب‌اپلیکیشن‌ها استفاده شود، قابل توسعه به صورت اپلیکیشن نیز خواهد بود که برای انتشار اولیه امتیاز بیشتری به وب‌اپلیکیشن داده می‌شود.
	\item
	اعضای تیم به طراحی و پیاده سازی به صورت وب‌اپلیکیشن تسلط وجود دارد.
	\item 
	با توجه به محدودیت زمانی موجود پیاده سازی در بستر وب ریسک انجام پروژه را کاهش می‌دهد.
\end{enumerate}

