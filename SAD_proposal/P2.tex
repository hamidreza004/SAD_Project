
%----------------------------------------------------------------------------------------
%	PART
%----------------------------------------------------------------------------------------


\chapter{
	پیش‌زمینه‌ی پروژه
}

%----------------------------------------------------------------------------------------
%	CHAPTER 2
%----------------------------------------------------------------------------------------

%\chapterimage{chapter_head_2.pdf} % Chapter heading image

\section{طرح مسئله}
سامانه‌های بهبود زندگی روزمره افراد در راستای صرفه‌جویی در یکی از موارد وقت، پول و انرژی بوجود آمده‌اند، سامانه دُنگ تلاش کرده که با تمرکز با  حداقل کردن هزینه زمانی کاربران و همچنین کمینه کردن گردش پول اضافه دو مورد اول را تا حد ممکن کاهش دهد و همچنین با ساده طراحی شدن حداکثری سامانه انرژی بسیار اندکی از کاربران برای ثبت اطلاعات بگیرد که بتواند در ماموریت خود موفق باشد. 

مشکلی که دُنگ در راستای حل آن کوشش می‌کند ایجاد رویکردی منصفانه در گروه‌های دوستی، کاری و خانواده است به صورتی که افراد از بودن کنار یک‌دیگر لذت ببرند و تا حد ممکن درگیر فرآیند‌های خشکی مانند پرداخت بدهی و ... نشوند.


\section{پیشینه‌ی منجرشونده به تعریف پروژه}% \lr{History leading to project request}} :-????????

یکی از بزرگ‌ترین چالش‌های گروه‌های مختلف دوستی و کاری در سرتاسر جهان مفهوم تسهیم هزینه‌ها با دوستان است، زیرا اغلب اوقات در یک گروه خاص فردی «مادر خرج» می‌شود و پس از آن دوستانش به سهو (یا حتی عمد!) فراموش می‌کنند که سهم خود را پرداخت کنند یا حتی آمار حساب پرداختی‌ها و بدهی‌ها را از دست می‌دهند.

تا کنون این مشکل با استفاده از روش‌های سنتی مانند نوشتن یا ذخیره کردن خرج‌ها و محاسبه دستی و زمان‌گیر میزان جابجایی‌های که لازم است انجام بگیرد، حل می‌شده که تمامی این راه‌حل‌ها بسیار زمان‌گیر و انرژی‌بر است و معمولا افراد حاضر نمی‌شوند که چنین هزینه‌ای را بدهند و چنین مسائل مالی خشک و ناراحت‌کننده‌ای را وارد فضاهای دوستی و خانوادگی خود بکنند.

از طرفی جز از سمت افرادی که معمولا پرداخت‌ها را انجام می‌دهند، برای افرادی که قرضی گرفته‌اند یا خرجی برایشان انجام شده هم سخت است که لیست تمامی پول‌هایی که باید در آینده پرداخت کنند را نگه‌ دارند، لذا در این بخش از محیط دوستانه و خرج‌ها هم خلاءی حس می‌شود که لازم است راه‌حلی ارائه شود که مشکل برطرف گردد.

از سویی دیگر هم گاهی اوقات دایره پرداختی‌ها آنقدر پیچیده می‌شود که میزان پول خرد و کارمزد بانک‌ها باعث می‌شود به طور کلی انتقال پول‌ها به صرفه در نظر نرسد، در حالی که می‌توانستیم با کمینه کردن تراکنش‌ها و تغییر شروع و مبدا پرداختی‌ها این میزان را کمتر کنیم و دغدغه کاربران را کاهش دهیم.

تمامی موارد بالا باعث شده‌اند که سامانه دُنگ بوجود بیاید، این سامانه به کاربران که گستره آن ها تمامی افراد جامعه هستند کمک می‌کند تا وضعیت مالی خود و روابط دوستانه خود را بهبود ببخشند.


\section{اهداف}
هدف اصلی این پروژه، طراحی، پیاده‌سازی و اجرای یک سامانه‌ی برخط است که کاربران مختلف به‌کمک آن بتوانند خرج‌های انجام شده را ثبت کرده و افراد مرتبط به آن‌ خرج‌ها را از میان دوستان خود در سامانه انتخاب کنند. این سامانه ویژگی‌های مختلفی را پوشش می‌دهد که از خرج‌های گروهی تا خرج‌های فرد به فرد را شامل می‌شود.

پس به طور کلی و خلاصه، این سامانه تلاش می‌کند تا اهداف زیر را برآورده سازد:
\begin{itemize}
	\item
	سامانه‌ای برای وارد کردن خرج‌های انجام شده با جزییات آن خرج‌ها و افراد درگیر آن باشد.
	\item 
	سامانه‌ای باشد که در صورت اقدام کاربر به اطلاع پیدا کردن از بدهی‌ها و قرض‌هایی که داده بتواند اطلاعات را به آن فرد ارائه کند.
	\item 
	بستری باشد که کاربران بتوانند با ایجاد گروه‌ها به سادگی به خرج‌های جمعی رسیدگی کنند.
	\item
	بتواند با انجام بهینه‌سازی تعداد گردش پول‌ها و میزان پول‌های خرد را کمتر کند.
\end{itemize}

\section{توصیف محصول}
محصول نهایی، سامانه دُنگ خواهد بود. همانطور که پیش‌تر هم اشاره کردیم، هدف نهایی این سامانه بهبود زندگی روزمره کاربران است.

در این سامانه تنها یک دسته کاربر عمومی وجود دارد، که پس از ایجاد حساب کاربری، وارد حساب کاربری خود شده و می‌تواند هزینه‌ها را به صورت گروهی (توانایی ساختن گروه با دوستان و همچنین اضافه کردن دوست به حساب کاربری وجود دارد) یا تکی وارد کند. حساب‌کاربری با ایمیل و شماره تلفن و هزینه‌ها با تاریخ و افراد مرتبط و عکس و موقعیت مکانی مشخص می‌شوند و کاربران می توانند با استفاده از ایمیل یا شماره تلفن فردی را به عنوان دوست خود اضافه کنند.

برای سادگی و کمتر کردن هزینه زمانی کاربران، ویژگی هزینه‌های گروهی نیز وجود دارد که می‌توان به صورت دقیق مشخص کرد که هر فرد چه درصدی از هزینه کلی را باید پرداخت کند یا اینکه باید هزینه به صورت مساوی میان همه تقسیم شود. همچنین تمامی کاربران می‌توانند با مراجعه به گزارش‌ها بدهی و قرض‌های کنونی و گذشته خود را مشاهده کنند و در صورت توان بدهی‌های خود را پرداخت کنند.

از طرفی سامانه برای کمتر کردن تعداد گردش‌های پولی، در صورتی که می‌توان با جابجایی پرداخت‌ها و حذف افراد میانی تعداد گردش پول‌ها را کمتر کرد در گروه‌های تشکیل شده بهینه‌سازی برای پرداخت‌ها انجام می‌دهد.




%\section{Paragraphs of Text}\index{Paragraphs of Text}
%
%\lipsum[1-7] % Dummy text
%
%%------------------------------------------------
%
%\section{Citation}\index{Citation}
%
%This statement requires citation \cite{article_key}; this one is more specific \cite[162]{book_key}.
%
%%------------------------------------------------
%
%\section{Lists}\index{Lists}
%
%Lists are useful to present information in a concise and/or ordered way\footnote{Footnote example...}.
%
%\subsection{Numbered List}\index{Lists!Numbered List}
%
%\begin{enumerate}
%	\item The first item
%	\item The second item
%	\item The third item
%\end{enumerate}
%
%\subsection{Bullet Points}\index{Lists!Bullet Points}
%
%\begin{itemize}
%	\item The first item
%	\item The second item
%	\item The third item
%\end{itemize}
%
%\subsection{Descriptions and Definitions}\index{Lists!Descriptions and Definitions}
%
%\begin{description}
%	\item[Name] Description
%	\item[Word] Definition
%	\item[Comment] Elaboration
%\end{description}
%
%%----------------------------------------------------------------------------------------
%%	CHAPTER 2
%%----------------------------------------------------------------------------------------
%
%\chapter{In-text Elements}
%
%\section{Theorems}\index{Theorems}
%
%This is an example of theorems.
%
%\subsection{Several equations}\index{Theorems!Several Equations}
%This is a theorem consisting of several equations.
%
%\begin{theorem}[Name of the theorem]
%	In $E=\mathbb{R}^n$ all norms are equivalent. It has the properties:
%	\begin{align}
%	& \big| ||\mathbf{x}|| - ||\mathbf{y}|| \big|\leq || \mathbf{x}- \mathbf{y}||\\
%	&  ||\sum_{i=1}^n\mathbf{x}_i||\leq \sum_{i=1}^n||\mathbf{x}_i||\quad\text{where $n$ is a finite integer}
%	\end{align}
%\end{theorem}
%
%\subsection{Single Line}\index{Theorems!Single Line}
%This is a theorem consisting of just one line.
%
%\begin{theorem}
%	A set $\mathcal{D}(G)$ in dense in $L^2(G)$, $|\cdot|_0$. 
%\end{theorem}
%
%%------------------------------------------------
%
%\section{Definitions}\index{Definitions}
%
%This is an example of a definition. A definition could be mathematical or it could define a concept.
%
%\begin{definition}[Definition name]
%	Given a vector space $E$, a norm on $E$ is an application, denoted $||\cdot||$, $E$ in $\mathbb{R}^+=[0,+\infty[$ such that:
%	\begin{align}
%	& ||\mathbf{x}||=0\ \Rightarrow\ \mathbf{x}=\mathbf{0}\\
%	& ||\lambda \mathbf{x}||=|\lambda|\cdot ||\mathbf{x}||\\
%	& ||\mathbf{x}+\mathbf{y}||\leq ||\mathbf{x}||+||\mathbf{y}||
%	\end{align}
%\end{definition}
%
%%------------------------------------------------
%
%\section{Notations}\index{Notations}
%
%\begin{notation}
%	Given an open subset $G$ of $\mathbb{R}^n$, the set of functions $\varphi$ are:
%	\begin{enumerate}
%		\item Bounded support $G$;
%		\item Infinitely differentiable;
%	\end{enumerate}
%	a vector space is denoted by $\mathcal{D}(G)$. 
%\end{notation}
%
%%------------------------------------------------
%
%\section{Remarks}\index{Remarks}
%
%This is an example of a remark.
%
%\begin{remark}
%	The concepts presented here are now in conventional employment in mathematics. Vector spaces are taken over the field $\mathbb{K}=\mathbb{R}$, however, established properties are easily extended to $\mathbb{K}=\mathbb{C}$.
%\end{remark}
%
%%------------------------------------------------
%
%\section{Corollaries}\index{Corollaries}
%
%This is an example of a corollary.
%
%\begin{corollary}[Corollary name]
%	The concepts presented here are now in conventional employment in mathematics. Vector spaces are taken over the field $\mathbb{K}=\mathbb{R}$, however, established properties are easily extended to $\mathbb{K}=\mathbb{C}$.
%\end{corollary}
%
%%------------------------------------------------
%
%\section{Propositions}\index{Propositions}
%
%This is an example of propositions.
%
%\subsection{Several equations}\index{Propositions!Several Equations}
%
%\begin{proposition}[Proposition name]
%	It has the properties:
%	\begin{align}
%	& \big| ||\mathbf{x}|| - ||\mathbf{y}|| \big|\leq || \mathbf{x}- \mathbf{y}||\\
%	&  ||\sum_{i=1}^n\mathbf{x}_i||\leq \sum_{i=1}^n||\mathbf{x}_i||\quad\text{where $n$ is a finite integer}
%	\end{align}
%\end{proposition}
%
%\subsection{Single Line}\index{Propositions!Single Line}
%
%\begin{proposition} 
%	Let $f,g\in L^2(G)$; if $\forall \varphi\in\mathcal{D}(G)$, $(f,\varphi)_0=(g,\varphi)_0$ then $f = g$. 
%\end{proposition}
%
%%------------------------------------------------
%
%\section{Examples}\index{Examples}
%
%This is an example of examples.
%
%\subsection{Equation and Text}\index{Examples!Equation and Text}
%
%\begin{example}
%	Let $G=\{x\in\mathbb{R}^2:|x|<3\}$ and denoted by: $x^0=(1,1)$; consider the function:
%	\begin{equation}
%	f(x)=\left\{\begin{aligned} & \mathrm{e}^{|x|} & & \text{si $|x-x^0|\leq 1/2$}\\
%	& 0 & & \text{si $|x-x^0|> 1/2$}\end{aligned}\right.
%	\end{equation}
%	The function $f$ has bounded support, we can take $A=\{x\in\mathbb{R}^2:|x-x^0|\leq 1/2+\epsilon\}$ for all $\epsilon\in\intoo{0}{5/2-\sqrt{2}}$.
%\end{example}
%
%\subsection{Paragraph of Text}\index{Examples!Paragraph of Text}
%
%\begin{example}[Example name]
%	\lipsum[2]
%\end{example}
%
%%------------------------------------------------
%
%\section{Exercises}\index{Exercises}
%
%This is an example of an exercise.
%
%\begin{exercise}
%	This is a good place to ask a question to test learning progress or further cement ideas into students' minds.
%\end{exercise}
%
%%------------------------------------------------
%
%\section{Problems}\index{Problems}
%
%\begin{problem}
%	What is the average airspeed velocity of an unladen swallow?
%\end{problem}
%
%%------------------------------------------------
%
%\section{Vocabulary}\index{Vocabulary}
%
%Define a word to improve a students' vocabulary.
%
%\begin{vocabulary}[Word]
%	Definition of word.
%\end{vocabulary}