
%----------------------------------------------------------------------------------------
%	PART
%----------------------------------------------------------------------------------------


\chapter{
	پیش‌زمینه‌ی پروژه
}

%----------------------------------------------------------------------------------------
%	CHAPTER 2
%----------------------------------------------------------------------------------------

%\chapterimage{chapter_head_2.pdf} % Chapter heading image

\section{طرح مسئله}
پروژه‌های سیستم‌های اطلاعاتی سه آغازکننده‌ی کلی دارند که عبارتند از، وجود یک مشکل، وجود یک فرصت، و یا بخش‌نامه یا چیزهایی از این دست.
در این مسئله، آغازکننده‌ی پروژه از جنس
\emph{وجود یک مشکل}
است. این مشکل هم همان عدم وجود یک سامانه‌ی اطلاعاتی مناسب، برای کاریابی دانشجویان و فارغ‌التحصیلان دانشگاه صعنتی شریف است.
در ادامه بیشتر پیرامون این موضوع، و این که چرا می‌بایستی این مشکل را حل‌کنیم و اینکه چرا اساسا این نکته به یک مشکل و مسئله تبدیل شده‌است می‌پردازیم.


\section{پیشینه‌ی منجرشونده به تعریف پروژه}% \lr{History leading to project request}} :-????????
در گذشته و حال، دانش‌جویان بدون استفاده از سامانه‌های اطلاعاتی و به‌طور مستقیم یا نهایتاً از طریق پوسترهای منتشرشده در سطح دانشگاه و یا از طریق دوستان و آشنایان، با موقعیت‌های کارآموزی و شغلی آشنا می‌شدند. با توجه به وقت و ارتباطات محدود دانش‌جویان، احتمال این‌که دانش‌جو بتواند مناسب‌ترین موقعیت ممکن برای خودش را پیدا کند، بسیار کم خواهد بود. هم‌چنین وقت‌گیر بودن این روند، باعث می‌شود دانش‌جویان دانشگاه صنعتی شریف که در زمینه‌های زیادی فعال هستند، لزوماً بهترین شغلی ممکن برای موقعیت خود را پیدا نکنند. همین‌طور تعداد شرکت‌هایی که در بازار کار هستند به مرور زمان در حال افزایش است و احتمال پیدا کردن بهترین موقعیت شغلی روز به روز کمتر می‌شود. 

از طرفی کارفرماها نیز در این روند متضرر می‌شوند زیرا متقابلاً آن‌ها نیز بهترین نیرو برای موقعیت‌های شغلی در سازمانشان پیدا نمی‌کنند. هم‌چنین تبلیغات در سطح دانشگاه یا از طرق مختلف، برای کارفرما هزینه‌بر است. 

حال اگر این کارفرمایان تازه شروع به‌کار کرده باشند و خیلی شناخته‌شده نباشند، احتمالاً در پیدا کردن نیروی مناسب به مشکلات زیادی برخواهند خورد. چرا که اکثر دانشجویان، برای پیدا کردن شغل به شرکت‌های بزرگتر مراجعه می‌کنند. و ممکن است مثلاً یک استارت‌آپ تازه راه‌اندازی‌شده که پتانسیل بالایی دارد، نتواند نیروی کار مناسب خود را پیدا کند. 

بنابراین این سامانه می‌تواند به دانشجویان یا فارغ‌التحصیلان شریف که به دنبال موقعیت مناسب خود می‌گردند، کمک شایانی کند. همین‌طور کارفرمایانی که به‌دنبال افراد خاصی از بین کارجویان هستند، می‌توانند با استفاده از این سامانه، با کم‌ترین هزینه بهترین نیروها را به شرکت خود جذب کنند. 


\section{اهداف}
هدف اصلی این پروژه، طراحی، پیاده‌سازی و اجرای یک سامانه‌ی برخط است که دانشجویان و فارغ‌التحصیلان دانشگاه صنعتی شریف به‌کمک آن بتوانند کسب‌و‌کار مورد نظر خود را از بین کسب‌و‌کارهایی که برای جذب این نیروها آمادگی خود را اعلام کرده‌اند، انتخاب کنند. 

پس به طور کلی و خلاصه، این سامانه قرار است اهداف زیر را برآورده سازد:
\begin{itemize}
	\item
	سیستمی خاص‌منظوره مختص کارجویانی از دانشجویان، و فارغ‌التحصیلان دانشگاه صعنتی شریف باشد.
	\item 
	بستری باشد که کارفرماها بتوانند آگهی‌های فرصت‌های شغلی خودشان را در سامانه وارد کنند و از این طریق کارجوها را استخدام کنند.
	\item 
	بستری باشد تا دانشجویان و فارغ‌التحصیلان جویای کار نیز بتوانند با مشاهده‌ی آگهی‌ها برای فرصت شغلی مورد نظرشان درخواست بدهند.
	\item
	بستری باشد تا هم کارفرماها و کارجویان بتوانند برای کارجو و شغل مناسبشان جست‌وجو کنند.
\end{itemize}

\section{توصیف محصول}
محصول و یا فرآورده‌ی نهایی این پروژه، سامانه‌ی «شریف‌کار» خواهد بود. که همان‌طور که در قسمت اهداف گفته شده‌است، هدف نهایی آن کاریابی بهتر و آسان‌تر برای دانش‌جویان و فارغ‌التحصیلان دانشگاه صنعتی شریف و تسهیل روند پیدا کردن شغل برای آن‌ها، و نیرو برای شرکت‌ها خواهد بود. 

در این سامانه، دو نوع کاربر عمومی خواهیم داشت: صاحبان کسب‌و‌کارها و دانشجویان/فارغ‌التحصیلان دانشگاه صنعتی شریف. 

برای بهره‌بری از سامانه، کارفرمایان ابتدا باید عضو سامانه شوند. سپس، با ورود به حساب کاربری خود، می‌توانند اطلاعات مربوط به کسب‌و‌کار خود و موقعیت‌های شغلی‌شان را روی سامانه قرار دهند. این اطلاعات شامل اطلاعات مربوط به شرکت مانند موقعیت جغرافیایی آن روی نقشه، شرایط کاری، مزایا و حقوق برای کار در آن شرکت می‌شود. همین‌طور نوع کار (کارآموزی، پاره‌وقت و یا دائمی)، پیش‌نیازهای فردی که استخدام می‌شود مانند مهارت‌های مورد انتظار، سابقه‌ی کاری، جنسیت، سن و ... را در سامانه نیز قرار داده‌ می‌شود.

دانشجویان/فارغ‌التحصیلان دانشگاه صنعتی شریف نیز با مراجعه به بخش مختص به خود، می‌توانند در سامانه حساب کاربری بسازند و سپس اطلاعات خود را در آن قرار دهند. یعنی اطلاعات فردی را در قالب رزومه‌ای بارگزاری کنند و مشخصات مربوط به شغل دلخواه خویش را مشخص نمایند. 

سپس دانشجویان/فارغ‌التحصیلان می‌توانند 

همچنین محصول شامل یک سامانه‌ی توصیه‌گر\LTRfootnote{recommender system}
خواهد بود.
این سامانه، بر اساس اطلاعات وارد شده توسط کارجویان و کارفرمایان، با بررسی شباهت‌ها بین نیازمندی و پیش‌نیازهای کارفرما و همین‌طور علاقه‌مندی و سوابق کارجو، می‌تواند به این دو گروه کاربر کمک کند. یعنی در صورتی که کارجویی توسط این سامانه برای موقعیت شغلی‌ای مناسب شناخته شده باشد، کارجو را به کارفرمای مورد نظر معرفی می‌کند. حال کارفرما این امکان را خواهد داشت که پس از بررسی رزومه‌ی کارجو، برای استخدام او اقدام نماید. همین‌طور این سامانه، به هر کارجو، آگهی‌های مرتبط با علاقه‌مندی‌ها و مهارت‌های او را پیشنهاد می‌دهد.  




%\section{Paragraphs of Text}\index{Paragraphs of Text}
%
%\lipsum[1-7] % Dummy text
%
%%------------------------------------------------
%
%\section{Citation}\index{Citation}
%
%This statement requires citation \cite{article_key}; this one is more specific \cite[162]{book_key}.
%
%%------------------------------------------------
%
%\section{Lists}\index{Lists}
%
%Lists are useful to present information in a concise and/or ordered way\footnote{Footnote example...}.
%
%\subsection{Numbered List}\index{Lists!Numbered List}
%
%\begin{enumerate}
%	\item The first item
%	\item The second item
%	\item The third item
%\end{enumerate}
%
%\subsection{Bullet Points}\index{Lists!Bullet Points}
%
%\begin{itemize}
%	\item The first item
%	\item The second item
%	\item The third item
%\end{itemize}
%
%\subsection{Descriptions and Definitions}\index{Lists!Descriptions and Definitions}
%
%\begin{description}
%	\item[Name] Description
%	\item[Word] Definition
%	\item[Comment] Elaboration
%\end{description}
%
%%----------------------------------------------------------------------------------------
%%	CHAPTER 2
%%----------------------------------------------------------------------------------------
%
%\chapter{In-text Elements}
%
%\section{Theorems}\index{Theorems}
%
%This is an example of theorems.
%
%\subsection{Several equations}\index{Theorems!Several Equations}
%This is a theorem consisting of several equations.
%
%\begin{theorem}[Name of the theorem]
%	In $E=\mathbb{R}^n$ all norms are equivalent. It has the properties:
%	\begin{align}
%	& \big| ||\mathbf{x}|| - ||\mathbf{y}|| \big|\leq || \mathbf{x}- \mathbf{y}||\\
%	&  ||\sum_{i=1}^n\mathbf{x}_i||\leq \sum_{i=1}^n||\mathbf{x}_i||\quad\text{where $n$ is a finite integer}
%	\end{align}
%\end{theorem}
%
%\subsection{Single Line}\index{Theorems!Single Line}
%This is a theorem consisting of just one line.
%
%\begin{theorem}
%	A set $\mathcal{D}(G)$ in dense in $L^2(G)$, $|\cdot|_0$. 
%\end{theorem}
%
%%------------------------------------------------
%
%\section{Definitions}\index{Definitions}
%
%This is an example of a definition. A definition could be mathematical or it could define a concept.
%
%\begin{definition}[Definition name]
%	Given a vector space $E$, a norm on $E$ is an application, denoted $||\cdot||$, $E$ in $\mathbb{R}^+=[0,+\infty[$ such that:
%	\begin{align}
%	& ||\mathbf{x}||=0\ \Rightarrow\ \mathbf{x}=\mathbf{0}\\
%	& ||\lambda \mathbf{x}||=|\lambda|\cdot ||\mathbf{x}||\\
%	& ||\mathbf{x}+\mathbf{y}||\leq ||\mathbf{x}||+||\mathbf{y}||
%	\end{align}
%\end{definition}
%
%%------------------------------------------------
%
%\section{Notations}\index{Notations}
%
%\begin{notation}
%	Given an open subset $G$ of $\mathbb{R}^n$, the set of functions $\varphi$ are:
%	\begin{enumerate}
%		\item Bounded support $G$;
%		\item Infinitely differentiable;
%	\end{enumerate}
%	a vector space is denoted by $\mathcal{D}(G)$. 
%\end{notation}
%
%%------------------------------------------------
%
%\section{Remarks}\index{Remarks}
%
%This is an example of a remark.
%
%\begin{remark}
%	The concepts presented here are now in conventional employment in mathematics. Vector spaces are taken over the field $\mathbb{K}=\mathbb{R}$, however, established properties are easily extended to $\mathbb{K}=\mathbb{C}$.
%\end{remark}
%
%%------------------------------------------------
%
%\section{Corollaries}\index{Corollaries}
%
%This is an example of a corollary.
%
%\begin{corollary}[Corollary name]
%	The concepts presented here are now in conventional employment in mathematics. Vector spaces are taken over the field $\mathbb{K}=\mathbb{R}$, however, established properties are easily extended to $\mathbb{K}=\mathbb{C}$.
%\end{corollary}
%
%%------------------------------------------------
%
%\section{Propositions}\index{Propositions}
%
%This is an example of propositions.
%
%\subsection{Several equations}\index{Propositions!Several Equations}
%
%\begin{proposition}[Proposition name]
%	It has the properties:
%	\begin{align}
%	& \big| ||\mathbf{x}|| - ||\mathbf{y}|| \big|\leq || \mathbf{x}- \mathbf{y}||\\
%	&  ||\sum_{i=1}^n\mathbf{x}_i||\leq \sum_{i=1}^n||\mathbf{x}_i||\quad\text{where $n$ is a finite integer}
%	\end{align}
%\end{proposition}
%
%\subsection{Single Line}\index{Propositions!Single Line}
%
%\begin{proposition} 
%	Let $f,g\in L^2(G)$; if $\forall \varphi\in\mathcal{D}(G)$, $(f,\varphi)_0=(g,\varphi)_0$ then $f = g$. 
%\end{proposition}
%
%%------------------------------------------------
%
%\section{Examples}\index{Examples}
%
%This is an example of examples.
%
%\subsection{Equation and Text}\index{Examples!Equation and Text}
%
%\begin{example}
%	Let $G=\{x\in\mathbb{R}^2:|x|<3\}$ and denoted by: $x^0=(1,1)$; consider the function:
%	\begin{equation}
%	f(x)=\left\{\begin{aligned} & \mathrm{e}^{|x|} & & \text{si $|x-x^0|\leq 1/2$}\\
%	& 0 & & \text{si $|x-x^0|> 1/2$}\end{aligned}\right.
%	\end{equation}
%	The function $f$ has bounded support, we can take $A=\{x\in\mathbb{R}^2:|x-x^0|\leq 1/2+\epsilon\}$ for all $\epsilon\in\intoo{0}{5/2-\sqrt{2}}$.
%\end{example}
%
%\subsection{Paragraph of Text}\index{Examples!Paragraph of Text}
%
%\begin{example}[Example name]
%	\lipsum[2]
%\end{example}
%
%%------------------------------------------------
%
%\section{Exercises}\index{Exercises}
%
%This is an example of an exercise.
%
%\begin{exercise}
%	This is a good place to ask a question to test learning progress or further cement ideas into students' minds.
%\end{exercise}
%
%%------------------------------------------------
%
%\section{Problems}\index{Problems}
%
%\begin{problem}
%	What is the average airspeed velocity of an unladen swallow?
%\end{problem}
%
%%------------------------------------------------
%
%\section{Vocabulary}\index{Vocabulary}
%
%Define a word to improve a students' vocabulary.
%
%\begin{vocabulary}[Word]
%	Definition of word.
%\end{vocabulary}